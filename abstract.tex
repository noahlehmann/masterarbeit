\clearpage
\newgeometry{top=2.4cm, bottom=2.4cm}
\pagestyle{empty}
\section*{Abstract}

Die Arbeit \emph{\mytitle{}} beschäftigt sich mit der intensiven Funktionsanalyse des \emph{Unitree Robotics Go1 Edu} und der
Einordnung potenzieller Einsatzmöglichkeit dessen in einem Hochschulumfeld.
Nach einer thematischen Einordnung der Arbeit werden Grundlagen zur Einordnung und Kategorisierung des Go1 in das Wissensgebiet
der Robotik erarbeitet.

Folgend wird der Aufbau des Roboters detailliert erschlossen und dokumentiert.
Hierfür werden zunächst die äußere Form beschrieben und die Bauteile benannt.
Daraufhin werden die intern verbauten Bauteile als weitere Referenz detailliert dokumentiert.
Anhand des Aufbaus der Interna wird die Kommunikation und der Aufbau des internen Netzwerks beschrieben.

Im folgenden Analyseteil wird die Nutzung dokumentiert, indem zunächst der Lieferumfang und das externe Zubehör erfasst wird, worauf zwei
Formen der Inbetriebnahme gezeigt werden, die vorgesehene Akkubetriebene Inbetriebnahme und die Inbetriebnahme
des Roboters mit einer externen Stromversorgung.
Die mitgelieferten mobile Anwendungen des Go1 werden in ihrem Funktionsumfang beschrieben.

Folgend wird im Hauptteil der Funktionsumfang des Roboters selbst beschrieben, wonach Erweiterungen des Umfangs mit
dem Ziel des autonomen Einsatzes erarbeitet werden.
Die beschriebenen Funktionen sind die Software Autostartfunktion, die Fernsteuerung, das lokale Netzwerk,
die Audiointerface Nutzung, die Steuerung der Kopfbeleuchtung, das Übertragen der Kamerabilder und das Batteriemanagement.

Abschließend wird in vier Schritten der autonome Einsatz des Go1 durch Erweiterungen des Funktionsumfangs vorbereitet.
Zuerst wird die Konnektivität des Roboters erweitert und ausfallsicherer gestaltet.
Hierfür wird für die lokale Verbindung des Roboters mit WLAN Access Points konfiguriert sowie der Einsatz einer
Mobilfunkkarte zur mobilen Verbindung erläutert und detailliert erarbeitet.
Eine Schwäche des Roboters im Einsatz wird behoben, indem das Batteriemanagement System um eine sichere Absicherung
des Gerätes bei niedrigem Akkustand erweitert wird.
Zur Vorbereitung der Fernsteuerung des Roboters wird eine Möglichkeit des Videostreamings anhand gängiger Protokolle
gezeigt, woraufhin die Fernsteuerung selbst implementiert wird.
Diese baut auf der stabilen Verbindung des Roboters über ein Netzwerk auf und baut auf den Erkenntnissen aus den Analyse-Kapiteln auf.

Zum Abschluss wird der Unitree Robotics Go1 Edu bewertet und eine Erweiterung des Roboters anhand der in der Arbeit
dokumentierten Erkenntnisse vorgestellt.
Die in mehreren Schritten und einzelnen Projekten zu erarbeitenden Erweiterungen ergeben kombiniert ein autonom
funktionierendes System, für das letztendlich nur sinnvolle Einsatzzwecke erforscht werden müssen.

\restoregeometry



\section{Einleitung}
\label{sec:einleitung}

% Robotik und KI wird immer populärer
In der jüngeren Vergangenheit haben Roboter und Automationen im industriellen und privaten Gebrauch immer stärker an Relevanz gewonnen.
\todo{Formulierung}
So soll laut \emph{Statista} die Anzahl der weltweit verwendeten Roboter \num{2027} im Vergleich zu diesem Jahr (\num{2023})
um fast \num{20}\% auf \num{4,9} Millionen Einheiten steigen.
Im Vergleich zu \num{2016} sind das sogar über \num{80}\%.\footcite{statista_robotics_market}
% durch effizientere Software, kleinere Hardware, mehr rechenpower, kleinere Akkus
% - Kostengünstig und leistungsfähige Sensoren
% - Fortschritte in KI, zulezt Sprachgebrauch
% - Miniaturisierung und kostensenkung der Komponenten in Mechanik und Elektronik
% - Breitentauglichkeit der Programmierung
Gründe hierfür sind unter anderen besonders die Entwicklung und Verfügbarkeit immer kostengünstigerer und leistungsfähigerer Sensoren,
die jüngere Entwicklungen im Bereich der \gls{ki}, die Miniaturisierung der benötigten Bauteile und letztlich auch
die anbahnende Breitentauglichkeit der Programmierung durch Sprachen wie \emph{Python}.
\todo{Nachweise}


Der stetige Wachstum der Anzahl der Roboter besonders im industriellen Bereich sowie im Bereich der Haushaltshelfer
lenkt die Aufmerksamkeit der Allgemeinheit immer weiter auf das Feld der Robotik. \todo{Belege}
Besonders in Verbindung mit \gls{ki} gewinnt das Feld weiter an Relevanz.
Die Aufmerksamkeit der Allgemeinheit aber auch der Industrie und der Forschung bewirkt, dass zunehmend neue Formen
der Roboter entwickelt und erforscht werden.
Einige Trends sind beispielsweise:

\begin{itemize}
    \item Roboterarten aufzählen
    \item Quadruped Roboter
\end{itemize}

Diese Arbeit wird sich mit der Entwicklung und dem Umgang von Quadruped Robotern der Firma \emph{Unitree} beschäftigen.









% ab hier alt
% Viele neue Formen und dadurch Anwendungsgebiete kristalisieren sich
% Viele der Formen werden noch nicht in voller effizienz genutzt

Durch die weite Verbreitung der Roboter und das erhöhte Interesse der Wirtschaft entstehen vermehrt neue Formen von Robotern.
Zu diesen gehören Iterationen der bekannten Formen, wie einarmige Industrieroboter, humanoide Roboter - auch Androide genannt - oder
Tierwesen nachempfundene Formen - zoomorphe Roboter, wie beispielsweise sechsbeinige Roboter.
Die hohe Entwicklungsrate und der rasante Fortschritt auf diesem Gebiet führt einerseits zu sehr fähigen Geräten,
deren Einsatzgebiete aber auf der anderen seite oftmals nicht oder kaum erforscht sind.
% Eine in letzter Zeit sehr bekannt gewordene Form ist die des vierbeinigen Roboters
% Die Arbeit beschäfftigt sich mit einem verbraucherorientierten Roboter und den möglichen einsatzfeldern
Unter anderem durch die Erfolge des Unternehmens \emph{Boston Dynamics} und besonders deren Roboter \emph{Spot} ist die zoomorphe Form
des Hundes in der breiten Masse der Anwender bekannt geworden.
Die folgende Arbeit setzt sich mit dem Umgang und der Weiterentwicklung solcher vierbeinigen Roboter auseinander.

% \subsection{Hintergrund und Kontext}\label{subsec:hintergrund-und-kontext}
% Im Rahmen des Ausbaus der Forschung in Robotik und KI hat FH Hof Quadruped Roboter gekauft
% Diese sind vergleichsweise günstig in Anschaffung und sehr flexibel einsetzbar
% Da Verbraucherorientiert muss erarbeitet werden, wie sie genutzt werden können
% Was die grenzen sind, wo die einsatzgebiete liegen

Im Zuge des Ausbaus der Robotik in der Lehre und Forschung der \emph{\gls{haw}} hat
die Fakultät Informatik der Hochschule zwei Roboter des Models \emph{Go1 Edu} der Marke \emph{Unitree Robotics} beschafft.
Im Vergleich zu einigen Alternativprodukten dieser und anderer Firmen ist der \gls{go1} sehr kostengünstig in der Anschaffung und besonders flexibel einsetzbar.
Die offenen Schnittstellen des \gls{go1} machen ihn ebenfalls für Forschungsinstitute besonders interessant.
Der genaue Funktionsumfang und die potenziellen Einsatzgebiete des Roboters werden im späteren Verlauf der Arbeit genauer erläutert.
Das folgende Kapitel grenzt den Umfang der Arbeit genauer ein.

% \subsection{Zielsetzung und Methodik}
\label{subsec:zielsetzung-und-methodik}
\todo{Ziel klarer definieren, was macht die arbeit??}
% Funktionserweiterung des Go1, um ihn sinnvoll nutzen zu können
% Detaillierte einführung in die Nutzung erster Teil, zweiter Teil erweiterung um zu integrieren


% Ziel:
% Die Arbeit beschäftigt sich weniger mit Robotik und KI an sich
% Mehr mit integration des Roboters in ein Hochschul-Umfeld
% IT- Infrastruktur, Rechenleistung outsourcen
% Software Robust halten

Trotz des Themas \emph{\mytitle} beschäftigt sich diese Arbeit weniger mit dem Thema \emph{Robotik} und
dem eng gekoppeltem Themenbereich der \gls{ki}.
Stattdessen liegt der Fokus auf der Integration und Funktionserweiterung des Roboters innerhalb eines
bestehenden IT-Ökosystems und der Umgebung, in dem dieses eingerichtet ist.
Das Umfeld ist in diesem Fall exemplarisch das der \gls{haw}.
Konkret besteht das Ökosystem der Hochschule aus folgenden Teilbereichen:

\begin{itemize}
    \item Studierende, Mitarbeiter und Besucher des Campuses
    \item Flächen und Gebäude
    \item IT-Infrastruktur inklusive des Rechenzentrums
    \item Lehr- und Forschungsinhalte
\end{itemize}

Ziel dieser Arbeit ist es, die Einsatzbereiche des \gls{go1} so zu testen, dass sie dem Ökosystem der \gls{haw} möglichst viel Nutzen bereiten.
\paragraph{Bewertungskriterien}
\todo{Wie wird Nutzen bewertet?}

% Methodik
% Einige Grundlagen klären:
% - Robotik, was benötigt in Arbeit?
% - KI, was ist, was benötigt, was relevant? etc Rechenleistung
% Zur Einleitung wird der Aufbau des Roboters
% Dann Nutzung und Inbetriebnahme
% - Einfache Nutzung
% Dann Interne Hardware Komponenten
% Überblick über die Komponenten
% Clustering der Komponenten in Anwendungsbereiche
% Erläuterung der Einsatzgebiete
% Verbindung aller Elemente (Netzwerk)

Das Ziel soll Schrittweise erreicht werden.
Anfangs werden einige Grundlagen erläutert und definiert.
So wird genauer auf Begriffe wie \emph{Robotik} und \emph{\gls{ki}}eingegangen, bevor diese im Laufe der Arbeit
weiter vorausgesetzt werden.
Zudem wird in diesem Schritt auch eingegrenzt, welche Teilbereiche dieser Fachgebiete relevant für den Verlauf der Arbeit sind.

Im zweiten Schritt wird der in dieser Arbeit referenzierte \gls{go1} genauer betrachtet.
Die einzelnen Bauteile werden aufgegliedert und ihre Funktion beschrieben.
Abschließend zu diesem Schritt soll noch die Zusammenarbeit der Komponenten erläutert werden.
Nachdem der \emph{Ist}-Stand des Roboters erklärt wurde, wird die im Lieferumfang ab Werk enthaltene Funktion gezeigt und dokumentiert.
Zur Inbetriebnahme des \gls{go1} benötigt es vorerst kein Wissen über die internen Steuerkomponenten.

Diese werden im nächsten Schritt genauer betrachtet.
Die Systemarchitektur und die Funktionen der einzelnen Hardware-Bauteile sollen genau dokumentiert werden.
Zudem wird hierbei auch die Kommunikation des Gesamtsystems in sich, als auch mit externen Komponenten wie Fernbedienungen
und Endgeräten dokumentiert.
\todo{In Gleiderung noch kommunikation einfügen oder in Protokollen abgedeckt?}
Hierzu werden auch die Limitierungen des Roboters im Werkszustand aufgelistet.

Als Nächstes wird darauf eingegangen, welche Schritte am Roboter getätigt werden können, um ihn sinnvoll
in das \gls{haw} Ökosystem zu integrieren.
Die einzelnen Möglichkeiten werden dann genauer aufgeschlüsselt und erläutert sowie bewertet.
Des Weiteren sollen dann die finalen Limitierungen des Roboters und mögliche zusätzliche Erweiterungen gezeigt werden.
Abschließend wird bewertet, inwiefern die Eingliederung eines \gls{go1} Roboters sinnvoll ist und welches Potenzial dieses Vorgehen birgt.

% \subsection{Eingrenzung der Arbeit} % Eingrenzung Arbeit
% Was wird nicht gemacht?
% Robotik low level
% KI im eigentlichen Sinne
Diese Arbeit ist im Allgemeinen auf die Integration in ein Ökosystem und die Funktionserweiterung des \gls{go1} in diesem Zug limitiert.
Das umfasst, dass Erweiterungen der Funktion im Bereich der Robotik und \gls{ki} auf der Hardware des Roboters selbst nur inkludiert sind, wenn
diese zweckdienlich und zwingend notwendig für eine Erweiterung im Sinne der oben genannten Integration ist.
\todo{Was wird nicht gemacht?}


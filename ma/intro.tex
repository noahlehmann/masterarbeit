\section{Einleitung}

% Robotik und KI wird immer populärer
% durch effizientere Software, kleinere Hardware, mehr rechenpower, kleinere Akkus

% Viele neue Formen und dadurch Anwendungsgebiete kristalisieren sich
% Viele der Formen werden noch nicht in voller effizienz genutzt

% Eine in letzter Zeit sehr bekannt gewordene Form ist die des Vierbeinigen Roboters
% Die Arbeit beschäfftigt sich mit einem verbraucherorientierten Roboter und den möglichen einsatzfeldern

\subsection{Hintergrund und Kontext}

% Im Rahmen des Ausbaus der Forschung in Robotik und KI hat FH Hof Quadruped Roboter gekauft
% Diese sind vergleichsweise günstig in Anschaffung und sehr flexibel einsetzbar
% Da Verbraucherorientiert muss erarbeitet werden, wie sie genutzt werden können
% Was die grenzen sind, wo die einsatzgebiete liegen

\subsection{Begriffsdefinitio? nötig}
% Aktuell redundant zur 2.1

\subsection{Zielsetzung und Methodik}

% Ziel:
% Die Arbeit beschäfftigt sich weniger mit Robotik und KI an sich
% Mehr mit integration des Roboters in ein Hochschul-Umfeld
% IT- Infrastruktur, Rechenleistung outsourcen
% Software Robust halten

% Methodik
% Einige Grundlagen klären:
% - Robotik, was benötigt in Arbeit?
% - KI, was ist, was benötigt, was relevant? etc Rechenleistung
% Zur Einleitung wird der Aufbau des Roboters
% Dann Nutzung und Inbetriebnahme
% - Einfache Nutzung
% Dann Interne Hardware Komponenten
% Überblick über die Komponenten
% Clustering der Komponenten in Anwendungsbereiche
% Erläuterung der Einsatzgebiete
% Verbindung aller Elemente (Netzwerk)

% Was wird nicht geklärt?

\subsection{Struktur der Arbeit} % Eingrenzung Arbeit

% Was wird nicht gemacht?


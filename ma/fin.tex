\clearpage

\section{Fazit}
\label{sec:fazit}

Dieses Kapitel dient als Abschluss und Zusammenfassung der erarbeiteten Erkenntnisse dieser Arbeit.
Zu Beginn werden die Ergebnisse dieser Arbeit zusammengefasst und bewertet.
Danach soll die Vorgehensweise und das Arbeitsmittel in Form des \gls{go1} bewertet werden.
In diesem Zug wird auch ein Fazit zur Integration dessen in ein Hochschul-Ökosystem geschlossen.
Abschließend soll ein Ausblick gestaltet werden, der die potenziell nächsten Schritte der Arbeit am Roboter in Aussicht stellt.

\subsection{Rückblick}
\label{subsec:ruckblick}

Diese Arbeit hatte besonders das Ziel, als Grundlage für die Arbeit mit dem \gls{go1} angesehen zu werden, um ihn dann in einem
Hochschulumfeld in der Forschung, Lehre und anderen Gebieten einsetzen zu können.
Hierfür wurde der Roboter in ein vereinfachtes Feld der Serviceroboter eingeordnet, wobei ebenfalls erläutert wurde, welche weiteren
Merkmale zutreffend sind.
Anschließend wurde der Aufbau und Umfang des GO1 in der Edu Version genau dokumentiert, sodass dieser Teil als Referenz für die
Einarbeitung in den Roboter verwendet werden kann.
Im Architekturbereich wurden die internen Komponenten genau analysiert.
Hierzu gehören auch die genauen Kenndaten der verbauten Recheneinheiten, deren Funktionen und besonders die Kommunikation
der Einheiten untereinander.

Anhand der gewonnenen Erkenntnisse über den Roboter konnte mit der Analyse der Funktionen begonnen werden.
Nach Dokumentation des Lieferumfangs und der diversen Möglichkeiten der Inbetriebnahme wurden diverse Grundfunktionen sowie
vereinzelt auch fortgeschrittene Funktionen des \gls{go1} dokumentiert und bewertet.
Viele dieser Funktionen werden zwar vom Hersteller beworben, leider jedoch nicht dokumentiert, weshalb die Arbeit
an dieser Stelle als detaillierte und besonders umfangreiche alternative Dokumentation angesehen werden kann.
Anzumerken ist hier der hohe Detailgrad der Untersuchungen bei Funktionen, die nicht oder nur unzureichend
dokumentiert oder nicht vollständig funktionsfähig sind.
Da nicht alle Funktionen des Roboters getestet wurden, wurde anschließend eine Liste der wichtigsten nicht bearbeiteten Funktionen
angefertigt, in der auch Referenzen zu einer anderen Arbeit gegeben sind, die sich mit großen Teilen dieser
Funktionen beschäftigt.

Abschließend wurde auf dem Wissen des Aufbaus, der Komponenten und der Funktionen des \gls{go1} eine Reihe neuer Funktionen
und Erweiterungen erarbeitet, welche im Allgemeinen alle das Ziel verfolgen, den Roboter aus der Ferne steuern zu können
und somit deutlich flexibler einsetzbar zu machen.
Ein weiteres Ziel der Funktionen ist es, die Grundlage für eine autonome Nutzung des Roboters zu schaffen, in dem man es
steuernder Software ermöglicht, auf die Daten des Roboters zuzugreifen, ohne direkt verbunden zu sein.

Zuletzt wird die Arbeit noch bewertet und ein Ausblick geschaffen, welcher in den folgenden Abschnitten erarbeitet wird.

\subsection{Einschätzung}
\label{subsec:einschatzung}

Die Arbeit am \gls{go1} ist eine meist ertragreiche, jedoch oftmals auch mühsame.
Einerseits sind die Präzision der Bauteile, besonders der mechanischen Bauteile der Motoren und Gelenkstrukturen, die Qualität
verbauten Komponenten, wie der drei NVIDIA Jetsons und der Raspberry Pi, sowie der Funktionsumfang von Teilen der
gelieferten Software beeindruckend, besonders in Relation zum Einstiegspreis des Roboters.
Jedoch zeigt die Ausstattung und Umsetzung besonders des Edu Models des GO1 klare Defizite, möglicherweise durch die kurze
Entwicklungszeit, welche sich anhand der verwendeten open-source Software und der Bekanntgabe des Verkaufs annähernd auf
zwei bis drei Jahre schätzen lässt.
Dieses Model ist auch bedeutend teurer in der Anschaffung, was die initiale Erwartung deutlich erhöht.

Positiv anzumerken sind die Öffnung aller Systeme für die Erweiterung durch neue Funktionen, die ausreichende Rechenkraft
der verbauten Einplatinencomputer und die umfangreiche Ausstattung an Sensorik.
Besonders die Verwendung von Open-Source Komponenten wie den Betriebssystemen, verwendeten nativen Bibliotheken und
Treibern für verbaute Zusatzkomponenten erleichtern die Untersuchung und Erweiterung der Funktionen, die der \gls{go1}
bereits ab Werk liefert.

Negativ anzumerken sind hingegen die unzureichende Dokumentation der Roboter, besonders der Unitree-Bibliotheken, sowie
das Verbergen der Implementierungen durch die ausschließliche Verwendung von Binärcode für native Funktionen.
Das macht es besonders bei der Suche nach Fehlern bei zu testenden Funktionen nahezu unmöglich, die Ursachen zu identifizieren,
ohne die Bestandssoftware detailliert zu untersuchen und teilweise zu Reverse-Engineering zurückzugreifen.

Eine einfache Integration des \gls{go1} in ein Hochschul-Ökosystem ist somit ohne angemessenen Aufwand nicht möglich.
Die initiale Dokumentation der Funktionen, das Testen und Bewerten dieser muss als Grundlage dieser Arbeit gesehen werden.
Um den Roboter dann in realen Situation an der Hochschule einsetzen zu können, kann dann auf die erarbeiteten Erkenntnisse
aufgebaut werden.
Für die Integration in die Lehre ist der Roboter hingegen ein wertvolles Gut.
Das Testen der Funktionen und das Erweitern dieser durch eigene Entwicklungen kann in verschiedenen Kenntnisständen der
Robotik durchgeführt werden.
So ist dieser Roboter durch die Nutzung bekannter Komponenten wie der NVIDIA Karten und des Raspberry Pis eine solide Plattform
für Anfänger auf dem Gebiet.
Arbeitet man jedoch intensiver mit dem Gerät, so bedarf es eines weiten Spektrums an Kenntnissen, unter anderen in den Bereichen
der Grundlageninformatik, hardwarenaher Programmierung und der Programmierung allgemein, Netzwerktechnik, Physik, Elektrotechnik,
Systemadministration und besonders der \gls{ki}.
Richtig eingesetzt ist der \gls{go1}, sowie ähnliche Geräte, eine sinnvolle Investition in eine Hochschule und der Lehre dieser.


\subsection{Ausblick}
\label{subsec:ausblick}

Zuletzt soll diese Arbeit noch als Referenz für zukünftige Projekte am \gls{go1} dienen.
Die Ausarbeitung des Funktionsumfangs des Roboters kann in mehreren Schritten erfolgen, welche in ihrer Komplexität
steigen und aufeinander aufbauen.
Die Erweiterungen haben das Ziel, den Roboter möglichste autonom verwenden zu können, ihm also eine Aufgabe zuzuweisen,
die er dann in allen Belangen ohne menschliche Unterstützung erledigen kann.

Der erste Schritt ist die dauerhafte Erreichbarkeit des Roboters für Entwickler und externe Server, unabhängig des Standortes
dessen.
Hierfür können die Erkenntnisse aus den Kapiteln \nameref{subsubsec:wifi} und \ref{subsubsec:gsm} verwendet werden, um
den Roboter per \gls{vpn} an einen Server zu verbinden, über den er nun dauerhaft und unabhängig der Art der Verbindung zum Internet
erreichbar ist.

Über diese Verbindung kann im zweiten Schritt eine Art Austausch der Logs und aktuellen Daten zur Telemetrie, des Standortes und
des \gls{bms} zwischen dem Roboter und einem fest installierten Server implementiert werden.
Dies ist nicht zwingend trivial, da hierfür das Format der Daten, das Intervall des Austauschs und die Art der Daten festgelegt werden
müssen.
Zudem muss das Protokoll zur Kommunikation und das Format der zu speichernden Daten erarbeitet werden.
Ein besonderes Augenmerk muss hierfür auf die Latenz der Übertragung gelegt werden, da diese im Laufe dieser Arbeit
oftmals ein Problem in der Effizienz der genutzten Funktionen darstellte.

Nach Austausch der relevanten Daten in einem effizienten, robusten und festgelegten Format kann an der Arbeit der manuellen
Fernsteuerung des Systems begonnen werden.
Auch hier muss der Fokus auf der Übertragungsgeschwindigkeit der Daten und der Zuverlässigkeit der Verbindung liegen.
Im Rahmen der Fernsteuerung kann ebenfalls an der effizienten Videoübertragung gearbeitet werden, da diese zur sicheren Steuerung
des Roboters ohne Sichtkontakt zu diesem zwingend nötig ist.
Auch die Nutzung der Ultraschalldaten kann hier in Betracht gezogen werden.

Als letzter Schritt zur Autonomie ist abschließend die Steuerung zu bearbeiten.
Denkbar sind hier zwei Szenarien, die besonders von den Erkenntnissen zur Qualität der Netzwerkverbindung aus den
vorigen Schritten abhängig sind.
Ist die Übertragung der Daten unzuverlässig, so sollte die Übertragungsmenge verringert werden und der Roboter
über eingebettete Software gesteuert werden.
Hier ist der Fokus auf die Effizienz der Implementierung zu legen, da die Recheneinheiten des Roboters nur
eine begrenzte Leistung liefern können.
Ist die Übertragungsrate ausreichend, so kann die Steuerung des Roboters auf ein externes System ausgelagert
werden.
Hier ist der Fokus auf die Übertragung der Daten gelegt.

Abschließend kann der \gls{go1} anhand der Erkenntnisse dieser Arbeit und der möglichen Erweiterungen auf
mögliche Einsatzzwecke erforscht werden, um den Übergang der akademischen Nutzung zu einer kommerziellen Nutzung
zu schaffen, die der Gesellschaft potenziell Nutzen bringt, was im Endeffekt das ultimative Ziel aller akademischen Arbeiten
ist, diese Arbeit mit dem Titel \emph{\mytitle} eingeschlossen.



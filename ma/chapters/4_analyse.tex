\section{Analyse des Roboters}
\label{sec:analyse-des-roboters}
Dieses und die folgenden Kapitel beschäftigen sich lediglich mit der Infrastruktur rund um den Roboter
und der Hardware und den Funktionen, die bereits im Roboter verbaut sind oder ergänzt werden können.
Die Funktionen rund um \gls{ml}, erweiterte Robotik, \todo{abk} LIDAR werden nicht behandelt.
Dieses Kapitel beschäftigt sich mit dem Ansatz der Analyse des Roboters.
Es wird gezeigt, wie die bestehenden Funktionen getestet und genutzt werden können, wie erkannt wird,
welche Funktionen bereits aktiviert sind und welche Teile der Soft- oder Hardware nicht aktiviert sind.
Zum Abschluss werden die bereits vorhanden Funktionen in dem Umfang, in dem sie ab Werk geliefert wurden,
gezeigt und erklärt.
Einige der Funktionen werden im späteren Verlauf der Arbeit auch erweitert oder verändert.
Die Dokumentation hierzu ist in Kapitel~\ref{sec:funktionserweiterungen-und-integration} zu finden.
Betroffene Funktionen werden hier im Kapitel explizit hervorgehoben.


\subsection{Vorgehensweise}
\label{subsec:vorgehensweise}
% Was mache ich, wenn ich was rausfinden möchte?
% Welche Ansätze für unbekannte Funktionen
% Meistens funktionierende Ansätze

\subsection{Funktionen}
\label{subsec:funktionen}
% Funktionen aufgelistet?
% Was wird hier eventuell explizit nicht bearbeitet?
% Referenz auf Jonas?

\subsubsection{Fernsteuerung}
% Fernsteuerung über Controller Bluetooth
% Fernsteuerung via Hinterherlaufen
%Was beachten, was einstellen

\subsubsection{Lokales Netzwerk}
% Wifi Wlan1 für Netzwerk
% Was wird verbreitet, wer nutzt, App etc

\subsubsection{Monitoring}
% App und Webschnittstelle
%

\subsubsection{Audio Interfaces}
% Mikrofone und Lautsprecher
% Wie kann ich eigene Dinge abspielen?
% Wo sind /dev/ registriert?
% Was ist installiert? (aplay)

\subsubsection{Video Streaming}
% nur möglichkeit, erweiterung in kap 6
% Websockets in Webseite und App
% Welche /dev/ sind vorhanden
% Wer kümmert sich? Pi oder Nanos etc

\subsubsection{Batterie Management}
% Cleveres System
% BMS über MQTT Monitoren
% Was wird angezeigt
% Wie werden Daten interpretiert?
% Wie herausgefunden?
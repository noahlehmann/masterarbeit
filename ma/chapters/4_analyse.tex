\section{Analyse des Roboters}
\label{sec:analyse-des-roboters}

Dieses und die folgenden Kapitel beschäftigen sich lediglich mit der Infrastruktur rund um den Roboter
und der Hardware und den Funktionen, die bereits im Roboter verbaut sind oder ergänzt werden können.
Die Funktionen rund um \gls{ml}, erweiterte Robotik und \gls{lidar} werden nicht behandelt.
Dieses Kapitel beschäftigt sich mit dem Ansatz der Analyse des Roboters.
Es wird gezeigt, wie die bestehenden Funktionen getestet und genutzt werden können, wie erkannt wird,
welche Funktionen bereits aktiviert sind und welche Teile der Soft- oder Hardware nicht aktiviert sind.
Zum Abschluss werden die bereits vorhanden Funktionen in dem Umfang, in dem sie ab Werk geliefert wurden,
gezeigt und erklärt.
Einige der Funktionen werden im späteren Verlauf der Arbeit auch erweitert oder verändert.
Die Dokumentation hierzu ist in Kapitel~\ref{sec:funktionserweiterungen-und-integration} zu finden.
Betroffene Funktionen werden hier im Kapitel explizit hervorgehoben.

Als Grundlage für die in diesem Kapitel vorausgesetzten Informationen wird Kapitel \ref{sec:roboterarchitektur-und-systemkomponenten}
referenziert.
Dieses sollte vor dem Weiterlesen betrachtet werden, falls noch kein umfangreiches Grundwissen über den Aufbau des \gls{go1}
vorhanden ist.
Ebenfalls zu beachten ist, dass es sich bei dem analysierten Roboter in den folgenden Kapiteln \ref{sec:analyse-des-roboters}
und \ref{sec:funktionserweiterungen-und-integration} um die Version \emph{GO1 Edu} handelt, welche deutlich umfangreichere
Ausstattung und Softwaremöglichkeiten mitliefert, als die weniger ausgestatteten Versionen des \gls{go1}.


\subsection{Inbetriebnahme}
\label{subsec:inbetriebnahme}

In diesem Kapitel wird die erste Inbetriebnahme des Roboters beschrieben.
Hierzu werden auch die Inhalte und Erweiterungen aufgelistet, welche im Lieferumfang für einen \gls{go1} Edu enthalten sind.
Anschließend soll der Betrieb durch den Akku oder ein Netzteil beschrieben werden.

\subsubsection{Lieferumfang}

Abbildung \ref{fig:lieferumfang} zeigt die im Lieferumfang enthaltene Transportbox aus Styropor und deren Inhalte, welche
gleichzeitig der gesamte Lieferumfang des \gls{go1} in der \emph{Edu} Version ist.

\begin{figure}[h]
    \frame{\includegraphics[width=\linewidth]{img/analyse/lieferumfang}}
    \caption{Ansicht der Transportbox und des Lieferumfangs}\label{fig:lieferumfang}
\end{figure}

\noindent Im Lieferumfang enthalten sind folgende Elemente:

\begin{enumerate}
    \item \gls{go1} inklusive Akku
    \item Kompakte Fernbedienung mit \emph{Follow-Me}-Funktion
    \item Netzteil für das Akku-Ladegerät
    \item Fernbedienung für An-/Ausschaltfunktion
    \item Ladegerät für die Fernbedienungen inklusive \gls{usb}-A auf \gls{usb}-C Kabel
    \item Vier Ersatzfüße inklusive integrierter Drucksensoren
    \item Batterieladegerät mit \gls{usb}-C Anschluss zur Akkuinspektion
    \item Fernbedienung
    \item Standfuß aus drei Teilen für die Ablage des \gls{go1}
    \begin{enumerate}
        \item Sockel
        \item Verlängerung
        \item Plattform inklusive Einkerbungen zum Ausbalancieren des aufliegenden \gls{go1}
    \end{enumerate}
\end{enumerate}

Sollte kein weiteres Zubehör konfiguriert sein, so sind an den Montagestellen auf der Oberseite des Roboters Gummienden
zum Schutz des Korpus bei Rotationen montiert.
Je nach bestelltem Zubehör -- wie beispielsweise eines professionellen \gls{lidar} Sensors oder eines
Roboterarmes am \gls{go1} -- sind Schienen an der Oberseite des Rumpfes montiert.
In diesem Fall werden die Gummienden sowie eine Schutzabdeckung für die Entwicklerports, die standardmäßig montiert ist,
in der Transportbox mitgeliefert.

\subsubsection{Mobile Inbetriebnahme}
\label{subsubsec:inbetriebnahme_akku}

Als mobile Inbetriebnahme wird hier die Inbetriebnahme des Roboters mit einer Stromversorgung durch den Akku bezeichnet.
Für die Inbetriebnahme auf diese Weise sind folgende Teile nötig:

\begin{requirements}
    \gls{go1}, Akku, Akkuladegerät inklusive Netzteil, Fernbedienung inklusive Ladegerät
\end{requirements}

\noindent Zur Vorbereitung müssen sowohl Akku als auch die Fernbedienung geladen werden.
Sind diese ausreichend geladen, kann der Akku eingesteckt und der \gls{go1} in die Ausgangsposition gebracht werden.
Hierfür müssen die vier Beine so rotiert werden, dass sowohl Fuß als auch Knie den Boden berühren.
Abbildung \ref{fig:ausgangsposition} zeigt die Ausgangsposition des Roboters.

\begin{figure}[h]
    \frame{\includegraphics[width=\linewidth]{img/analyse/ausgangsposition}}
    \caption{Ausgangsposition des Roboters}\label{fig:ausgangsposition}
\end{figure}

Durch einfaches Drücken auf dem Knopf über der Ladeanzeige des Akkus wird der Ladestand durch die vier \glspl{led}
angezeigt.
Durch ein sofortiges weiteres Drücken und Halten des Knopfes startet der Roboter.
Dies wird über ein kurzes serielles Aufblinken aller vier \glspl{led} signalisiert.
Zudem sind die Lüfter deutlich zu hören.
Durch dasselbe Verfahren -- Drücken gefolgt von zweitem Drücken und Halten des Knopfes neben der Ladeanzeige auf
der Unterseite der Fernbedienung wird auch diese angeschaltet.
Ein einmaliges akustisches Signal ertönt beim Anschalten.
Die Fernbedienung verbindet sich automatisch mit dem Roboter.
Im Werkszustand steht der Roboter nach dem Einschalten nach etwa \num{70} - \num{80} Sekunden auf.
Der Roboter kann nun über die Fernbedienung gesteuert werden.

Zum Ausschalten des \gls{go1} sollte dieser in eine liegende Position (\emph{Prone}-State) und anschließend in den \emph{Damping}-State gebracht werden.
Das wird durch die Tastenkombinationen \texttt{L2+A} und \texttt{L2+B} erreicht.
Danach kann der Akku wie beim Anschalten durch Drücken und erneutes Drücken und Halten ausgeschalten werden.
Die \glspl{led} signalisieren das erfolgreiche Ausschalten und die Lüfter schalten sich aus.
Auch die Fernbedienung kann durch dieses Vorgehen ausgeschalten werden.
Hier signalisieren drei kurze akustische Signale das erfolgreiche Ausschalten.

\subsubsection{Stationäre Inbetriebnahme}
\label{subsubsec:inbetriebnahme_netzteil}

Im Gegensatz zur mobilen Inbetriebnahme wird hier die Inbetriebnahme des Roboters durch eine stetige Stromversorgung durch ein Netzteil
bezeichnet.
Hierfür müssen einige Vorbereitungen getroffen werden, da im Lieferumfang kein Netzteil für den Betrieb des \gls{go1}
enthalten ist.

\begin{requirements}
    \gls{go1}, Netzteil für Akkuladegerät\newline
    \textbf{Zusätzlich:} Hohlstecker M \num{5,5}/\num{2,1} mm auf Schraubklemme, XT30-M Stecker mit ausreichend (> \num{10} cm)
    Verkabelung.
    Alternativ: XT30-M auf Hohlstecker M \num{5,5}/\num{2,1} mm Adapter.
\end{requirements}

Vergleicht man Ausgangsspannung und maximale Stromstärke des verbauten Akkus mit denselben Werten des Netzteils für das
Akkuladegerät, so erkennt man, dass das Ladegerät die passende Ausgangsspannung und Leistung vorweist, um den \gls{go1}
über den in Kapitel \ref{subsubsec:recheneinheiten} Abbildung \ref{fig:vogelperspektive} beschriebenen \emph{XT-30} Port
zu betreiben.
Da das Netzteil lediglich einen Hohlstecker zum Anschluss bietet, muss hierfür jedoch in Eigenarbeit ein Adapter auf XT-30
hergestellt werden.
Herbei sind die in den Ressourcen beschriebenen Bauteile nötig.
Die Schraubverbindung ist zwar optional und kann durch eine
Lötstelle oder ein vorgefertigtes Teil ersetzt werden, sie erleichtert jedoch die Beschaffung und den Zusammenbau des Adapters.
Zu beachten ist hierbei nur die korrekte Abmantelung der Kabel für die Schraubklemme und die Orientierung der Kabel in der
Klemme.
Abbildung \ref{fig:xt30} zeigt eine beispielhafte Umsetzung des Beschriebenen und die korrekte Orientierung des XT-30 Steckers.

\begin{figure}[h]
    \frame{\includegraphics[width=\linewidth]{img/analyse/xt30}}
    \caption{XT-30 auf Hohlstecker Verkabelung und XT-30 Orientierung}\label{fig:xt30}
\end{figure}

Aus Sicherheitsgründen sollte der sogenannte \emph{Sportmodus} des \gls{go1} deaktiviert werden, da sich dieser sonst nach
Start des Roboters aufsteht, was die Verkabelung lösen könnte.
Zudem ist die maximale Ausgangsleistung des Netzteils nicht hoch genug, um den Roboter dauerhaft inklusive der Motoren
zu betreiben.
Der Betrieb wird deshalb stationär genannt.
Für das Deaktivieren des Sportmodus muss sich zunächst auf den Raspberry Pi verbunden werden.
Im Home-Ordner des Nutzers \emph{pi} befindet sich der Ordner \texttt{Unitree/autostart/triggerSport} mit der Datei
\texttt{trig\-ger\-Sport.sh}.
Diese muss lediglich umbenannt werden, beispielsweise zu \texttt{trig\-ger\-Sport.dis\-a\-bled.sh}.

\begin{lstlisting}[label=lst:disable-triggersport]
pi@raspberrypi:~ $ cd Unitree/autostart/triggerSport/
pi@raspberrypi:~/Unitree/autostart/triggerSport $ ls
build  log  triggerSport.sh  version.txt
pi@raspberrypi:~/Unitree/autostart/triggerSport $ mv triggerSport.sh triggerSport.disabled.sh
pi@raspberrypi:~/Unitree/autostart/triggerSport $ ls
build  log  triggerSport.disabled.sh  version.txt
\end{lstlisting}

\noindent Beim Neustart des Roboters wird nun nicht automatisch in den Sportmodus geschaltet und die Motoren bewegen sich nicht.
Weiteres zur \emph{Autostart}-Funktion des \gls{go1} in Kapitel \ref{subsubsec:software-autostart}.

Zuletzt kann das Netzteil inklusive des Adapters in den XT-30 Port auf dem Rücken des Roboters gesteckt werden.
Ein erfolgreicher Start des Roboters kann über das akustische Wahrnehmen der Lüfter geprüft werden.
Zum Ausschalten muss der Stecker lediglich entfernt und der Roboter somit vom Strom getrennt werden.
Hierfür muss nur sichergestellt sein, dass alle manuellen Operationen auf den Recheneinheiten vollendet sind, damit diese
nicht korrumpiert werden.

\subsubsection{Grafische Anwendungen}
\label{subsubsec:anwendungen}

Nach Inbetriebnahme des \gls{go1} stehen dem Nutzer zwei grafische Anwendungen zur Verfügung, mit denen er Funktionen
des Roboters wie Monitoring, Steuerung und Simulationen ausführen kann.
Dieses Kapitel soll kurz die erste Einrichtung beziehungsweise das Öffnen der Anwendungen dokumentieren.

\myparagraph{Webinterface}

Der Raspberry Pi des \gls{go1} startet zum Systemstart automatisch einen Webserver mit einer Webseite, die diverse Funktionen
zugänglich macht.
Für das Aufrufen der Webseite muss man sich im selben Netzwerk wie der \gls{go1} befinden.
Weitere Informationen hierzu sind in den Kapiteln \ref{subsec:netzwerk} und \ref{subsubsec:lokales-netzwerk} beschrieben.
Auf dem \gls{http}-Port \num{80} des Pis\footnote{Erreichbar auf \texttt{192.168.123.161} oder \texttt{192.168.12.1} im eigenen WLAN-Hotspot (Siehe \ref{subsubsec:lokales-netzwerk})}
ist nun die Seite aufrufbar.
In Browsern reicht hierfür lediglich die Eingabe der \gls{ip} Adresse, der Port muss nicht definiert sein.
Abbildung \ref{fig:website} zeigt eine Bildschirmaufnahme des Webinterfaces.

\begin{figure}[h]
    \frame{\includegraphics[width=\linewidth]{img/analyse/website}}
    \caption{Screenshot des Webinterfaces}\label{fig:website}
\end{figure}

\myparagraph{Mobile App}

Unitree Robotics hat als Ergänzung zum Webinterface ebenfalls eine mobile Anwendung für die Plattformen \emph{iOS} und
\emph{Android} veröffentlicht.
Diese kann fertig gebaut von der Entwicklerwebseite heruntergeladen werden und muss auf den jeweiligen Plattformen als
fertige App aus unbekannter Quelle installiert werden\footcite{unitree_app_download}.
Hierfür sind in der Regel die Entwickleroptionen zu aktivieren.
Da die Installationsprozesse für die Plattformen und Endgeräte unterschiedlich sein können, wird in dieser Arbeit nicht
darauf eingegangen.
\footnote{Weitere Informationen zur Installation von Beta-Apps für iOS auf \url{https://testflight.apple.com/join/KraKgqam}
und für Android auf \url{https://www.heise.de/tipps-tricks/Externe-Apps-APK-Dateien-bei-Android-installieren-so-klappt-s-3714330.html}}
Abbildung \ref{fig:android} zeigt zwei Bildschirmaufnahmen der mobilen Anwendung.

\begin{figure}[h]
    \frame{\includegraphics[width=\linewidth]{img/analyse/android}}
    \caption{Screenshot der mobilen Anwendung}\label{fig:android}
\end{figure}

\noindent Für die Verbindung zum Roboter muss sich lediglich mit dem Netzwerk des Roboters verbunden werden.
Die \gls{ip} Adresse des Roboters muss dann noch in den Einstellungen hinterlegt werden.
Danach können die Informationen vom \gls{go1} über die Anwendung eingesehen werden.
\subsection{Funktionen}
\label{subsec:funktionen}

Im folgenden Kapitel werden einige der im Werkszustand mitgelieferten Funktionen des \gls{go1} erklärt und vorgeführt.
Einige der Funktionen sind bereits freigeschaltet und bedürfen keiner bis weniger Konfiguration.
Andere sind nicht freigeschaltet oder kaum dokumentiert, weshalb diese detaillierter beschrieben werden.
Teile der Funktionen sind nicht im Umfang der Arbeit erfasst worden.
Alle bekannten, nicht erfassten Funktionen werden am Ende des Kapitels aufgezählt.
\subsubsection{Software Autostart}
\label{subsubsec:software-autostart}

Unitree Robotics hat bei der Implementierung der Startsequenzen ihrer Softwarekomponenten eine einfache, für den Nutzer
zugängliche Autostart Funktion der Betriebssysteme des Raspberry Pis und der NVIDIA Jetsons verwendet.
Auf den Ubuntu Systemen wird das Paket \texttt{gnome\allowbreak -start\-up-\allowbreak app\-li\-cations} verwendet.
Die Installation des Paketes kann durch den Befehl \texttt{apt list --installed | grep gnome-\allowbreak start\-up-\allowbreak app\-li\-cations}
geprüft werden.
Anzumerken ist, dass die Funktionalität von der Desktop-Umgebung \emph{Gnome} bereitgestellt wird, nicht vom Betriebssystem
der Recheneinheiten.
Die auf \gls{lxde} basierende Desktop-Umgebung \gls{pixel}, die auf dem Raspberry Pi installiert ist, liefert eine andere
Möglichkeit der Autostart-Konfiguration.
Dennoch ist die Einrichtung identisch zur Bibliothek \texttt{gnome-\allowbreak start\-up-\allowbreak app\-li\-cations}, welche aber nicht auf dem System
installiert ist.
Eine offizielle Dokumentation für die Autostart-Funktion des Pis ist deshalb nicht verfügbar, es ist lediglich ersichtlich,
dass die Funktion dem auf dem Open-Source-Projekt \emph{FreeDesktop} beschriebenen Standard \emph{Autostart Of Applications During Startup}
entspricht\footcite{freedesktop_autostart}.
Die Funktion wird in diesem Kapitel exemplarisch am Raspberry Pi gezeigt.

\myparagraph{Implementierung}

Zu Prüfung der Implementierung muss sich vorerst auf den Raspberry Pi verbunden werden.
Hier wird zunächst geprüft, welcher Benutzer auf dem System nach einem Boot automatisch eingeloggt wird, damit die
Desktop-Umgebung die passenden Programme startet.
Erwartet wird hier der Benutzer \emph{pi}, was bei der Prüfung bestätigt wird.

\begin{lstlisting}
pi@raspberrypi:~ $ cat /etc/lightdm/lightdm.conf  | grep autologin-user | grep -v \#
autologin-user=pi
\end{lstlisting}

\noindent Der im FreeDesktop Standard definierte Ordner \texttt{/home/\allowbreak pi/\allowbreak .config/\allowbreak autostart/} enthält lediglich eine Datei des Typs
\texttt{.desktop}.
Dateien mit dieser Endung werden von der Desktop-Umgebung verwendet, um Programme zu starten und in der Oberfläche
gegebenenfalls weitere Informationen wie Bilder und Beschreibungen anzuzeigen.

\begin{lstlisting}
pi@raspberrypi:~ $ ls /home/pi/.config/autostart/
unitree.desktop
pi@raspberrypi:~ $ cat /home/pi/.config/autostart/unitree.desktop
[Desktop Entry]
Name=unitree
Comment=unitree autostart
Exec=bash /home/pi/UnitreeUpgrade/start.sh
Terminal=false
Type=Application
Categories=System;Utility;Archiving;
StartupNotify=false
NoDisplay=true
\end{lstlisting}

\noindent Der Wert des Feldes \texttt{Exec} wird nach dem Boot-Vorgang und dem Einloggen des Benutzers \emph{pi} als
Kommandozeilenbefehl interpretiert und ausgeführt.
Ein Blick auf die Zeilen \num{6} und \num{7} des ausgeführten Skripts verweise lediglich auf ein weiteres Skript.

\lstinputlisting[firstline=6,lastline=7,numbers=left,xleftmargin=2em,framexleftmargin=1.5em,firstnumber=6,caption={[Inhalt der Autostartdatei \texttt{/home/pi/UnitreeUpgrade/start.sh}]}]{listing/start.sh}

\noindent Das Skript \texttt{/home/\allowbreak pi/\allowbreak Unitree/\allowbreak autostart/\allowbreak update.sh}
legt eine neue Log-Datei an und initiiert dann den Autostart Prozess.

\begin{lstlisting}[numbers=left,xleftmargin=\dimexpr2.5em-1pt,framexleftmargin=2em,firstnumber=13]
for dir in `cat .startlist.sh`
do
  if [[ $dir = \#* ]] ; then
    echo $dir': skipped' >>/home/unitree/Unitree/autostart/.startlog
  else
    cd $dir
    echo $dir':'`sed -n '1p' version.txt` >> ${scriptPath}.detailedVersion
    ./$dir.sh
    sleep 3
  fi
  cd $scriptPath
done
\end{lstlisting}

\noindent Der Inhalt der Datei \texttt{.startlist.sh} ist eine durch Zeilenumbrüche getrennte Liste aller Ordner, welche
zum Autostart nach ausführbaren Skripten durchsucht werden.
In Zeile \num{13} wird über alle diese Ordner iteriert.
Zeile \num{15} prüft, ob der aktuelle Ordner in der Liste auskommentiert wurde und überspringt diesen gegebenenfalls.
In Zeile \num{18} wird andernfalls in den Ordner navigiert und in Zeile \num{20} jenes Skript in dem Ordner ausgeführt,
das denselben Namen wie der Ordner selbst inklusive der Dateiendung \texttt{.sh} hat.

Zur Erweiterung der Autostart-Funktion oder dem Hinzufügen eigener Prozesse nach Systemstart muss somit lediglich ein
Ordner im Pfad \texttt{/home/\allowbreak pi/\allowbreak Unitree/\allowbreak auto\-start/} angelegt werden, in dem eine
ausführbare Datei mit dem Namen des Ordners inklusive der Dateiendung \texttt{.sh} liegt.
Weitere Informationen zu dieser Vorgehensweise und ein Beispiel der Erweiterung werden in Kapitel \ref{subsec:bms-monitor}
gezeigt.

\subsubsection{Fernsteuerung}
\label{subsubsec:fernsteuerung}

Der \gls{go1} lässt sich auf vier verschiedene Arten fernsteuern:

\begin{itemize}
    \item Fernbedienung
    \item App
    \item Webinterface
    \item Folgefunktion
\end{itemize}

\noindent Dieses Kapitel beschreibt alle vier Möglichkeiten kurz und zeigt diverse Limitierungen der einzelnen Umsetzungen auf.
Für alle vier Steuermöglichkeiten muss der Roboter angeschaltet und der Sportmodus aktiviert sein.
Sollte der Sportmodus nicht aktiviert sein und eine Verbindung auf den Raspberry Pi nicht möglich oder erwünscht sein,
so kann dieser über die gekoppelte Fernbedienung und der Tastenkombination \texttt{L2+START} aktiviert werden.
Der \gls{go1} muss sich hierfür in der Ausgangsposition wie in Kapitel \ref{subsubsec:inbetriebnahme_akku} beschrieben befinden.

\myparagraph{Fernbedienung}

Der Lieferumfang des \gls{go1} umfasst zwei physische Fernbedienungen, mit denen der Roboter gesteuert werden kann.
Die Hauptfernbedienung besitzt zwei sogenannte Joysticks, welche die Position und Bewegung Roboter in verschiedenen Achsen
manipulieren können.
Die zweite Fernbedienung, von Unitree \emph{Label Controller} genannt, besitzt lediglich ein Joystick, welches die Bewegung
nach vorne, hinten, links und rechts steuern kann.
Sie dient ebenfalls als Sender für die \emph{Folge}-Funktion.
Abbildung \ref{fig:controller} zeigt links die Hauptfernbedienung und rechts den Label-Controller.

\begin{figure}[h]
    \frame{\includegraphics[width=\linewidth]{img/analyse/controller}}
    \caption{Hauptfernbedienung (links) und Label-Controller (rechts)}\label{fig:controller}
\end{figure}

Die einfache Bedienung der Hauptfernbedienung ist bereits nach dem Anschalten des Roboters nach Kapitel \ref{subsubsec:inbetriebnahme_akku} möglich.
Hierbei koppelt sich die Fernbedienung automatisch mit dem Roboter.
Dokumentation über die Art der Verbindung ist nicht auffindbar, es wird jedoch vermutet, dass dies nicht über Bluetooth,
sondern über ein anderes Protokoll geschieht und sich die Fernbedienung mit der \gls{mcu} statt mit dem Raspberry
Pi verbindet, welcher auch über Bluetooth verfügen würde.
Es beseht jedoch die Möglichkeit, die Fernbedienung via Bluetooth mit einem Smartphone zu koppeln.
Hierfür muss zur Verifikation der Kopplung der Pin \texttt{1234} verwendet werden.
Der Kopplungsprozess ist unter den verschiedenen Herstellern unterschiedlich und wird deshalb hier nicht weiter dokumentiert.
Nach erfolgreicher Kopplung kann über die mobile App\footnote{Siehe Kapitel \ref{subsubsec:anwendungen}} die Fernbedienung verbunden werden.
Über die Einstellungen und den Menüpunkt \texttt{Peripherals > Bluetooth Gamepad} kann in der Liste \texttt{Gamepad List}
die Fernbedienung mit der übereinstimmenden Seriennummer ausgewählt werden.
Diese ist ab Werk auf den Fernbedienungen gelabelt.
Abbildung \ref{fig:controller-app} zeigt die Bildschirmaufnahmen der relevanten Menüpunkte der App.

\begin{figure}[h]
    \frame{\includegraphics[width=\linewidth]{img/analyse/controller-app}}
    \caption{App-Menüpunkte \texttt{Peripherals > Bluetooth Gamepad} und \texttt{Gamepad List}}\label{fig:controller-app}
\end{figure}

\noindent Nun besteht die Möglichkeit, den \gls{go1} über die Fernbedienung fernzusteuern, egal in welchem Netzwerk er sich befindet.
Es ist lediglich notwendig, dass sich der Roboter und das Handy im selben Netzwerk befinden.
Weitere Informationen zur Netzwerkerweiterung sind in Kapitel \ref{sec:funktionserweiterungen-und-integration} dokumentiert.

\myparagraph{App}

Der \gls{go1} lässt sich ebenfalls durch die mobile Anwendung steuern, ohne sie vorher mit der Hauptfernbedienung verbunden zu haben.
Hierfür muss der Hauptmenüpunkt \texttt{Vision} ausgewählt werden.
Danach muss im rechten oberen Eck die Steuerung in der Ansicht der Kameras aktiviert werden.
Wie auf Abbildung \ref{fig:app-controller} gezeigt, kann dann der Modus im rechten unteren Eck des Bildschirmes gewählt werden.
Wählt man hier einen der Laufmodi aus, so lässt sich der \gls{go1} mit den beiden dargestellten Steuereinheiten bewegen.
Links stellt die linke Steuereinheit der Hauptfernbedienung dar, rechts die rechte Einheit.

\begin{figure}[h]
    \frame{\includegraphics[width=\linewidth]{img/analyse/app-controller}}
    \caption{Bildschirmaufnahme des App-Controllers}\label{fig:app-controller}
\end{figure}

\myparagraph{Webinterface}

Die Webseite, die auf dem Raspberry Pi gehostet wird, bietet im Menü \texttt{Vision} die Möglichkeit, im rechten
oberen Eck des Bildschirmes die Steuerung des Roboters zu aktivieren.
Nach Aktivierung werden dem Nutzer, wie auf Abbildung \ref{fig:web-controller} dargestellt, zwei Steuerelemente dargestellt.
Diese können links mit den Tasten \texttt{W-A-S-D} und rechts mit den Tasten \texttt{\textuparrow -\textleftarrow -\textdownarrow -\textrightarrow}
gesteuert.
Die linke Steuereinheit entspricht der linken Seite der Hauptfernbedienung, die rechte dementsprechend der rechten Seite.

\begin{figure}[h]
    \frame{\includegraphics[width=\linewidth]{img/analyse/web-controller}}
    \caption{Bildschirmaufnahme des Web-Controllers}\label{fig:web-controller}
\end{figure}

\myparagraph{Folgefunktion}

Der sogenannte Label-Controller lässt sich genauso wie die Hauptfernbedienung über die Tastenkombination Drücken und
erneutes Drücken und Halten des \texttt{POW}-Knopfes anschalten.
Mit dem Joystick lässt sich der Roboter vereinfacht steuern.
Die eigentliche Funktion des Label-Controllers ist jedoch die von Unitree Robotics beworbene \emph{Follow Me} Funktion.
Der Controller kommuniziert ständig mit dem Roboter und tauscht Informationen zur angenäherten Position des \gls{go1} relativ
zum Label-Controller aus.
Dies lässt sich in der mobilen App beobachten.
Hierfür muss über die Einstellungen auf den Pfad \texttt{Peripherals > Track Tag} navigiert werden.
Abbildung \ref{fig:follow-me} zeigt die beiden Bildschirmaufnahmen.

\begin{figure}[h]
    \frame{\includegraphics[width=\linewidth]{img/analyse/follow-me}}
    \caption{App-Menüpunkte \texttt{Peripherals > Track Tag} und die Tracking-Übersicht}\label{fig:follow-me}
\end{figure}

Um den Roboter zum Folgen zu bringen, muss der Wert der Option \texttt{Follow} angetippt werden.
Dieser sollte nun von \texttt{OFF} auf \texttt{ON} wechseln.
Danach bewegt sich der Hund immer relativ zum Label-Controller.
Durch die Tasten des Label-Controllers lässt sich das Verhalten minimal anpassen.
Tabelle \ref{tab:label-controller-tasten} fasst die Tastenfunktionen zusammen.

\begin{table}[h]
    \centering
    \begin{tabularx}{\textwidth}{|c|X|}
        \hline
        \textbf{Taste} & \multicolumn{1}{c|}{\textbf{Funktion}} \\ \hline
        POW & \begin{tabular}[c]{@{}l@{}}\num{1} Sekunde Drücken \textrightarrow{} Aufrichten nach Sturz\\ \num{2} mal kurz Drücken \textrightarrow{} Wechsel der Standmodi\\ Stehen \textrightarrow{} Legen \textrightarrow{} Motoren deaktivieren \textrightarrow{} Stehen\end{tabular} \\ \hline
        MOD & \begin{tabular}[c]{@{}l@{}}Kurzes Drücken \textrightarrow{} Folgen deaktivieren\\ \num{2} mal kurz Drücken \textrightarrow{} Wechsel der Modi\\ Langsames Folgen (\num{1,5} m/s) \textrightarrow{} Schnelles Folgen (\num{3} m/s)\end{tabular} \\ \hline
        A & Wechsel Ausweichmodus (nach Test nicht Funktionsfähig) \\ \hline
        B & \begin{tabular}[c]{@{}l@{}}Kurzes Drücken \textrightarrow{} Rotation gegen Uhrzeigersinn um etwa 6\textdegree \\ \num{2} mal kurz Drücken \textrightarrow{} Reset der Rotation auf Standardwert\end{tabular} \\ \hline
    \end{tabularx}\caption{Zusammenfassung der Tastenfunktionen des Label Controllers\label{tab:label-controller-tasten}}
\end{table}

\noindent Anzumerken ist, dass der Label-Controller nicht per Bluetooth mit dem Handy verbunden werden kann.

\subsubsection{Lokales Netzwerk}
\label{subsubsec:lokales-netzwerk}
% Wifi Wlan1 für Netzwerk
% Was wird verbreitet, wer nutzt, App etc

Der Raspberry Pi des \gls{go1} nutzt eines seiner Netzwerk-Schnittstellen, um ein \gls{wlan} zu publizieren.
Ein Blick auf das Autostartmodul \texttt{configNetwork} zeigt, dass hierfür das Interface \texttt{wlan1} verwendet wird.

\lstinputlisting[language=Bash,numbers=left,xleftmargin=\dimexpr2.5em-1pt,framexleftmargin=2em,firstline=35,lastline=38,firstnumber=35,caption={[Konfiguration des hostapd in \texttt{configNetwork.sh}]}]{listing/configNetwork.sh}

\noindent Um weitere Informationen Konfiguration des \gls{wlan} zu finden, muss die \gls{hostapd} Konfiguration in
\texttt{/etc/\allowbreak hostapd/\allowbreak hostapd\allowbreak .conf} betrachtet werden.
Der Service \texttt{\gls{hostapd}} ist ein für Nutzer eines Systems verfügbarer Service für diverse Access Points und
Authentifizierungsserver\footcite{hostapd-doc}.

\lstinputlisting[language=Bash,numbers=left,xleftmargin=\dimexpr2.5em-1pt,framexleftmargin=2em,firstline=18,lastline=22,firstnumber=18,caption={[Konfiguration des \texttt{hostapd}]}]{listing/hostapd.conf}

\noindent Hier ist ersichtlich, dass die \gls{ssid} des \gls{wlan} Access Points der Seriennummer des jeweiligen \gls{go1} entspricht.
Diese ist auf diversen Teilen des Roboters gekennzeichnet.
Das Standardpasswort ist \texttt{00000000} und kann in der Konfigurationsdatei ebenfalls geändert werden.
Nach Neustart des Interfaces \texttt{wlan1} tritt dieses neue Passwort in Kraft.

Ebenfalls hilfreich kann die Funktion einer versteckten \gls{ssid} sein, da der Access Point des \gls{go1} sonst in unmittelbarer
Nähe zum Roboter für jedermann einsichtig wäre, was ein potenzielles Sicherheitsrisiko im regulären Betrieb darstellt.
Hierfür muss lediglich die Zeile \texttt{ignore\_\allowbreak broadcast\_\allowbreak ssid=1} am Ende der Datei hinzugefügt werden.
Die Einstellung bewirkt, dass die Publizierung des \gls{wlan} keine \gls{ssid} bekannt gibt und Verbindungsanfragen ohne eine Angabe der
vollständigen \gls{ssid} ignoriert werden.


\subsubsection{Audio Interfaces}
\label{subsubsec:audio-interfaces}
% Mikrofone und Lautsprecher
% Wie kann ich eigene Dinge abspielen?
% Wo sind /dev/ registriert?
% Was ist installiert? (aplay)

% todo über app wiedergabe
\subsubsection{Kopfbeleuchtung}
\label{subsubsec:led}

Die \gls{led}-Reihen an den beiden Außenseiten des Kopfes sind, wie in Kapitel \ref{sec:roboterarchitektur-und-systemkomponenten}
beschrieben, am Jetson Nano des Kopfes angeschlossen.
Prüft man dort die Funktionen, die über den Autostart gesteuert werden, so fallen zwei Ordner auf -- \texttt{faceLightServer/}
und \texttt{faceLightMqtt/}.
Leider sind die Funktionalitäten beider Skripte als Binärdateien abgelegt und somit nicht ohne weiteres auslesbar.
Auch die Dokumentation des Herstellers weist keinerlei Informationen zu den \gls{led}-Reihen auf.
Der Name des zweiten Ordners weist jedoch auf eine mögliche Funktionalität der Steuerung über MQTT hin.
Um dies zu bestätigen, lässt sich ein MQTT-Explorer verwenden.
Dieser registriert sich als Client beim MQTT-Broker und schreibt alle Nachrichten und veröffentlichten Topics mit.
Mehr zum Thema MQTT gibt es auf der offiziellen Dokumentation des Standards\footnote{https://mqtt.org/}, Beispiele zur Nutzung
des Standards am Roboter in Kapitel \ref{sec:funktionserweiterungen-und-integration}.

Zur Nutzung des MQTT Explorers wird die \gls{ip}-Adresse des Brokers und der Port, auf dem dieser veröffentlicht wird, benötigt.
Für die Adresse kommen nur die \glspl{ip} \texttt{192.168.123.13}, \texttt{...14}, \texttt{...15} und \texttt{...161}
infrage.
Es kann mit der Bibliothek \emph{Nmap} geprüft werden, ob der MQTT-Standardport \texttt{1883} auf einer der registrierten \glspl{ip}
im Netz \texttt{192.168.123.0/24} geöffnet ist.

\begin{lstlisting}
nmap -sS -O -p1883 192.168.123.0/24
\end{lstlisting}

\noindent Die Ausgabe zeigt, dass die beiden \glspl{ip}
\texttt{192\allowbreak .168\allowbreak .123\allowbreak .15} und \texttt{192\allowbreak .168\allowbreak .123\allowbreak .161} den MQTT Port offen haben.
Über den MQTT Explorer wird zunächst der NVIDIA Jetson Xavier NX als Broker getestet.
Die Verbindung funktioniert, jedoch werden weder verfügbare Topics noch Messages ausgegeben.
Ein Test auf dem Raspberry Pi mit der \gls{ip} \texttt{192.168.123.161} zeigt die Ausgabe wie in Abbildung \ref{fig:mqtt-explorer}
dargestellt.

\begin{figure}[h]
    \frame{\includegraphics[width=\linewidth]{img/analyse/mqtt-explorer-no-facelight}}
    \caption{Ausgabe eines MQTT Explorers in Verbindung mit dem Raspberry Pi als Broker}\label{fig:mqtt-explorer}
\end{figure}

In den angezeigten Topics sind aktuell noch keine Informationen zu den \glspl{led} zu erkennen.
Dies kann daran liegen, dass das zu dem Topic noch keine Nachrichten versendet wurden, was eine plausible Schlussfolgerung
ist, da die Kopflichter des \gls{go1} standardmäßig ausgeschalten sind.
Ändert man dies nun durch die in der mobilen Anwendung gegebenen Funktion unter dem Menüpunkt \texttt{Preferences}, so
sieht man auch das Topic \texttt{face\_light/color} im MQTT-Explorer, wie in Abbildung \ref{fig:app-mqtt-facelight} dargestellt.

\begin{figure}[h]
    \frame{\includegraphics[width=\linewidth]{img/analyse/app-mqtt-facelight}}
    \caption{MQTT Explorer mit Topic \texttt{face\_light/color} (links) und App-Funktion (rechts)}\label{fig:app-mqtt-facelight}
\end{figure}

Die Message Payload kann beispielsweise über das Commandline-Tool \emph{Mosquitto-Client} ausgegeben werden.
Hierfür muss folgender Befehl genutzt werden, um die binäre Payload lesbar auszugeben.
Der genutzte Rechner muss sich hierfür im Netzwerk des \gls{go1} befinden.

\begin{lstlisting}
mosquitto_sub -h 192.168.123.161 -t face_light/color -F %x
ff0000
00ff00
0000ff
\end{lstlisting}

\noindent Die drei Werte in den letzten der Zeilen der Ausgabe sind formatierte Message-Payloads des Topics \texttt{face\_light/color}
und wurden ausgegeben, als in der mobilen Anwendung die drei Farben \emph{Rot}, \emph{Grün} und \emph{Blau} in eben dieser
Reihenfolge eingestellt wurden.
Somit ist herleitbar, dass die \glspl{led} des Roboters über den \emph{RGB}-Farbraum konfigurierbar sind.
Die Mischung aus Rot, Grün und Blau kann hier pro Farbe mit einem Wert von \num{0} - \num{255} eingestellt werden.

Über den Befehl \texttt{mosquitto\_pub} kann nun auch die Farbe des Roboters geändert werden.
Hierfür muss nur das Versenden der Message-Payload in binärer Form beachtet werden.
Folgender Befehl stellt das Kopflicht des \gls{go1} auf Rot um.

\begin{lstlisting}
echo -ne "\xFF\x00\x00" | mosquitto_pub -h 192.168.123.161 -t face_light/color -s
\end{lstlisting}

\noindent Es ist ebenfalls denkbar, die \glspl{led} des Roboters direkt über die \texttt{CP210x UART Bridge} zu steuern, die in Kapitel
\ref{par:nano-kopf} erwähnt wurde.
Dies wurde im Rahmen dieser Arbeit jedoch nicht umgesetzt und kann in Zukunft noch dokumentiert werden.

\subsubsection{Video Streaming}
\label{subsubsec:video-streaming}
% nur möglichkeit, erweiterung in kap 6
% Websockets in Webseite und App
% Welche /dev/ sind vorhanden
% Wer kümmert sich? Pi oder Nanos etc

Der \gls{go1} bietet mit seinen fünf Kameras die Möglichkeit, Bilder seiner Umgebung zu übertragen und es den Nutzern so
zu ermöglichen, den Roboter aus der Entfernung zu steuern.
Die Positionierung, Verteilung und die Mounting-Points der Kameras innerhalb des Roboters, den Recheneinheiten und den
Betriebssystemen wurde bereits in Kapitel \ref{subsec:hardware-architektur} geschildert.
Kurz zusammengefasst sind im Kopf des Roboters zwei Kameras positioniert, nach vorne und nach unten gerichtet.
Beide sind am Jetson Nano innerhalb des Kopfes verbunden.
Die beiden Außenseiten des Rumpfes sind mit zwei Kameras bestückt, die mit dem Jetson Nano m Rumpf des \gls{go1} verbunden sind.
Die letzte Kamera an der Unterseite des Rumpfes ist mit dem Jetson Xavier NX verbunden.
Am Beispiel der nach vorne gerichteten Kamera im Kopf des \gls{go1} soll in diesem Kapitel kurz erläutert werden,
wie auf die Kameras zugegriffen werden kann und wie man von einem verbundenen Rechner außerhalb des Roboters auf die Bilder
zugreifen kann.
Die dargestellte Anleitung ist für alle anderen Kameras bis auf etwaige Mounting-Points und \gls{ip}-Adressen identisch.

\myparagraph{Zugriff auf Kamerabilder}

Um die Kameras des \gls{go1} nutzen zu können, müssen zuerst alle Prozesse gestoppt werden, die die Geräte selbst blockieren.
Geprüft werden kann dies über den Befehl \texttt{fuser -vm /dev/\allowbreak video1}, hier muss nach der Ausgabe nach den Zeilen gesucht werden,
die als \texttt{ACCESS}-Flag den Wert \texttt{m} für \texttt{memory mapped files} haben.
Die Prozesse können dann mit deren Namen beendet werden.

\begin{lstlisting}
pkill -f point_cloud_nod
pkill -f example_point
\end{lstlisting}

\noindent Für den Zugriff auf die Daten der Kamera über den Mount-Point \texttt{/dev/\allowbreak video1} kann das Paket \texttt{ffmpeg}
genutzt werden, das auf allen Ubuntu Systemen des \gls{go1} vorinstalliert ist.
Folgender Befehl inklusive Erläuterung zu den Optionen kann genutzt werden, um per \texttt{ffmpeg} einen Videostream
über \gls{rtsp} auf einen Streaming-Server zu starten.

\begin{lstlisting}
ffmpeg -nostdin \                  # Keine Interaktion
    -f video4linux2 \              # Input Format
    -i /dev/video1 \               # Input-URL
    -vcodec libx264 \              # Video Kodierung (h264)
    -preset:v ultrafast \          # Kodierungsgeschwindigkeit
    -tune zerolatency \            # Keine H264 B-Frames
    -framerate 5 \                 # Framerate
    -f rtsp \                      # Output Format
    rtsp://<ip:8554|port>/<stream> # Output File/URL
\end{lstlisting}

\noindent Weitere Details zur Ausführung und dem Streaming über einen Server werden in Kapitel \ref{sec:funktionserweiterungen-und-integration}
behandelt.
Das Kamerabild ist beim Streaming über \texttt{ffmpeg} in keiner Form verarbeitet und sieht wie in Abbildung \ref{fig:kamera-bild}
dargestellt aus.

\begin{figure}[h]
    \frame{\includegraphics[width=\linewidth]{img/analyse/kamera-bild}}
    \caption{Kamerabild des Go1}\label{fig:kamera-bild}
\end{figure}

Eine Verarbeitung und Verbindung der beiden \emph{Fischaugen} der Kamerabilder ist im Roboter bereits mit der Bibliothek \emph{OpenCV}
umgesetzt.
Auch die in Kapitel \ref{subsubsec:ressourcen} genannte \emph{Unitree-Camera-SDK} nutzt OpenCV in der Implementierung.
Im Rahmen dieser Arbeit wird jedoch nicht auf die erweiterten Funktionen durch den Einsatz von OpenCV eingegangen.
Hierfür kann die Arbeit \citetitle{jonas}\footcite{jonas} zurate gezogen werden.
\subsubsection{Batterie Management}
\label{subsubsec:batterie-management}
% Cleveres System
% BMS über MQTT Monitoren
% Was wird angezeigt
% Wie werden Daten interpretiert?
% Wie herausgefunden?

Der Herstellerdokumentation und Werbung ist an einigen Stellen zu entnehmen, dass im \gls{go1} ein intelligentes \gls{bms} verbaut ist.
Dieses ermöglicht es Nutzern, in Echtzeit Informationen zum Stand der Batterie abzugreifen und gegebenenfalls auf die Informationen
zu reagieren.
In der Herstellerdokumentation des Roboters ist nicht dokumentiert, wie die Daten erfasst oder interpretiert werden können,
es wird lediglich auf die beiden Übersichten in der mobilen Anwendung und der Webseite verwiesen, die in Abbildung \ref{fig:bms-app-web}
dargestellt werden.

\begin{figure}[h]
    \frame{\includegraphics[width=\linewidth]{img/analyse/bms-app-web}}
    \caption{Batterieinformationen in der Webseite (links) und App (rechts)}\label{fig:bms-app-web}
\end{figure}

Die Ausgabe des MQTT-Explorers aus Kapitel \ref{subsubsec:led} beinhaltet ein Topic namens \texttt{bms/\allowbreak state}.
Auch die Prüfung der Webseite über die Entwicklertools innerhalb moderner Browser weist auf die Bereitstellung der
\gls{bms} Daten über MQTT hin.
Der Dateipfad \texttt{src/\allowbreak plugins/\allowbreak mqtt/\allowbreak receivers/\allowbreak bms\allowbreak Receivers\allowbreak .ts}
und das konfigurierte MQTT-Topic \texttt{bms\allowbreak /state} bestätigen dies.
Gibt man nun die Message-Payloads des Topics aus, so erhält man folgendes Ausgabeformat.

\begin{lstlisting}
mosquitto_sub -h 192.168.123.161 -t bms/state -F %x
0104011bf5e8ffff0f001e1e1f23800da00d20002000a00da00da00d20002000800d
\end{lstlisting}

Man kann über die Entwicklertools die Struktur der Daten zurückverfolgen.
Zeile \num{14} der Datei \texttt{bmsReceivers.ts} zeigt die Umwandlung der Message-Payload aus einem \texttt{Byte\allowbreak Buffer} in ein \texttt{Uint8Array}.
Laut der JavaScript-Dokumentation ist ein \texttt{Uint8Array} ein Array aus \num{8}-bit unsigned little Endian Integer\footcite{uint8array}.
Somit können je zwei Ziffern der hexadezimalen Ausgabe des \gls{bms} als ein Wert des Arrays interpretiert werden.
Die Zeilen \num{14} bis \num{17} und Zeile \num{20} in Listing \ref{lst:bms-reverse} zeigen die Umwandlungen der \num{8}-bit Integer.

\begin{lstlisting}[numbers=left,xleftmargin=\dimexpr2.5em-1pt,framexleftmargin=2em,firstnumber=14,label={lst:bms-reverse},caption=Inhalt der Webpack Datei \texttt{bmsReceivers.ts}]
const uint8s = new Uint8Array(message);
data.bms.version = uint8s[0] + "." + uint8s[1];
data.bms.status = uint8s[2];
data.bms.soc = uint8s[3];
data.bms.current = dataView.getInt32(4, true);
data.bms.cycle = dataView.getUint16(8, true);
data.bms.temps = [uint8s[10],uint8s[11],uint8s[12],uint8s[13]];
for (let i = 0; i < 10; i++) {
  data.bms.cellVoltages[i] = dataView.getUint16(14+i*2, true);
}
data.bms.voltage = data.bms.cellVoltages.reduce((a,c) => a+c);
\end{lstlisting}

\noindent Die Dokumentation der Klasse \texttt{DataView} zeigt, das die Funktionen \texttt{get\allowbreak Int32()} und \texttt{get\- Uint16()}
folgenden Syntax haben\footcite{dataview}.

\begin{lstlisting}
getInt32(byteOffset, littleEndian)
getUint16(byteOffset, littleEndian)
\end{lstlisting}

Folglich wird in Zeile \num{18} ein \num{4}-Byte Integer ab dem fünften Byte der Message-Payload ausgelesen, in den Zeile
\num{19} ein \num{2}-Byte Integer ab dem neunten Byte und zehn weitere ab dem fünfzehnten Byte der Payload gelesen.
Die zehn letzten \num{2}-Byte Integer werden zu einer Gesamtzahl addiert.
Durch die Auswertungen der Webseite ergibt sich folgender Überblick über das Format der Payload.
Das verwendete Beispiel ist die oben gezeigte Ausgabe des Befehls \texttt{mosquitto\_\allowbreak sub}.

\begin{table}[h]
    \centering
    \begin{tabularx}{\textwidth}{|X|X|X|X|XX|}
        \hline
        \textbf{Byte} & \textbf{Format} & \textbf{Inhalt} & \textbf{Beispiel} & \multicolumn{2}{c|}{\textbf{Konvertierung}} \\ \hline
        0-1 & 2 x uint8 & Version & 01, 04 & \multicolumn{2}{c|}{v1.4} \\ \hline
        2 & uint8 & Status & 01 & \multicolumn{2}{c|}{1} \\ \hline
        3 & uint8 & State of Charge & 1b & \multicolumn{2}{c|}{27 \%} \\ \hline
        4-7 & int32 & Strom & f5e8ffff & \multicolumn{2}{c|}{-5899 mA} \\ \hline
        8-9 & uint16 & Zyklus & 0f00 & \multicolumn{2}{c|}{15} \\ \hline
        10-13 & 4 x uint8 & Temperaturen & \begin{tabular}[c]{@{}c@{}}1e, 1e,\\ 1f, 23\end{tabular} & \multicolumn{2}{c|}{\begin{tabular}[c]{@{}c@{}}30 \textdegree C, 30 \textdegree C\\ 31 \textdegree C, 35 \textdegree C\end{tabular}} \\ \hline
        14-33 & 10 x uint16 & Zell-Spannung & \begin{tabular}[c]{@{}c@{}}800d, a00d,\\ 2000, 2000,\\ a00d, a00d,\\ a00d, 2000,\\ 2000, 800d\end{tabular} & \multicolumn{1}{c|}{\begin{tabular}[c]{@{}c@{}}3456 mV, 3488 mV,\\ 32 mV, 32 mV,\\ 3488 mV, 3488 mV,\\ 3488 mV, 32 mV,\\ 32 mV, 3456 mV\end{tabular}} & \begin{tabular}[c]{@{}c@{}}\textbf{Summe:}\\ 20,992 V\end{tabular} \\ \hline
    \end{tabularx}
    \label{tab:bms-format-matrix}\caption{Übersicht der BMS Message Payload}
\end{table}

Im Kapitel \ref{sec:funktionserweiterungen-und-integration} wird gezeigt, wie die Informationen des \gls{bms} sinnvoll
ausgelesen und verwertet werden können.



\subsection{Weitere Funktionen}
\label{subsec:weitere-funktionen}

Neben den in dieser Arbeit dokumentierten Funktionen bringt der \gls{go1} weitere Funktionen mit, die hier kurz
aufgelistet und erläutert werden sollen.

\begin{itemize}
    \item \textbf{\gls{ros}:}
    Auf einigen der Recheneinheiten des Roboters ist \gls{ros} installiert, was für umfangreiche Funktionserweiterungen
    im Bereich der Robotik genutzt werden kann.
    \item \textbf{\gls{slam}:}
    Der Roboter verfügt über die mobile Anwendung und einige vorinstallierte Bibliotheken die Funktion, durch die verbaute
    Sensorik ein Bild der Umgebung zu mappen und sich in diesem zu lokalisieren.
    \item \textbf{OpenCV Kameraauswertung:}
    Die mitgelieferte Bibliothek \emph{Unitree-Camera-SDK} nutzt OpenCV zur Auswertung der Kamerabilder.
    Durch OpenCV können die Fischaugen der Kameras kombiniert in ein Bild zusammengefasst, die Tiefenmessung der Kamera
    genutzt und die Bilder ausgewertet werden.
    \item \textbf{Ultraschallsensorik:}
    Die Ultraschallsensoren können ebenfalls über eine offizielle Bibliothek ausgelesen werden, um auf Bewegungen und Hürden
    im Umfeld des Roboters zu reagieren.
    \item \textbf{Bewegungssteuerung:}
    Die Motoren des \gls{go1} können in zwei Modi programmatisch gesteuert werden, den High und Low Level Modi.
\end{itemize}

\noindent Die Funktionen \gls{ros}, \gls{slam}, OpenCV und die Nutzung der Ultraschallsensoren werden in der Arbeit
\citetitle{jonas} aufgegriffen und dokumentiert\footcite{jonas}.
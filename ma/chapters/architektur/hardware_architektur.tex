\subsection{Hardware Architektur}
\label{subsec:hardware-architektur}

Im folgenden Kapitel wird der Aufbau des intern verbauten Hardware-Systems und die Funktion der einzelnen Bauteile
dessen dargestellt.
Hierfür wird vorerst ein kurzer Überblick über die Komponenten und ihren Zusammenhang geschaffen, was sich in Teilen
mit der Ausführung im vorigen Kapitel \ref{subsubsec:recheneinheiten} überschneidet.
Mit dem geschaffenen Überblick werden die Einzelteile des System genauer betrachtet und dokumentiert.
Dies soll über die einfache Dokumentation des Herstellers deutlich hinaus gehen.
Abschließend wird die interne Kommunikation der Bauteile in sich und mit externen Komponenten beschrieben.

\subsubsection{Überblick}
\label{subsubsec:ueberblick_hardware}

Als Einstieg in den Überblick soll veranschaulichend Abbildung \ref{fig:hardware_ueberblick} dienen.
Sie zeigt die einzelnen Komponenten und ihre Verbindungen untereinander und zu den erweiterten Systemen wie dem \gls{bms}
oder der Sensorik.

\begin{figure}[h]
    \frame{\includegraphics[width=\linewidth]{img/hardware_architektur/ueberblick}}
    \caption{Überblick über die interne Architektur des Go1}\label{fig:hardware_ueberblick}
\end{figure}

Als Grundlage für sowohl die interne Hardware als auch die mechanischen Komponenten dienen die \gls{mcu} und das \gls{bms}.
Beide werden vom Hersteller nicht für den Zugriff freigeschaltet und können indirekt verwendet werden.
So können beispielsweise die Daten der Motoren und deren Steuerung aktuell nur über Bibliotheken gelesen und manipuliert werden.
Auch die Daten des \gls{bms} sind nur lesend verfügbar.
Zentral zu allen Komponenten steht ein Switch, welcher diese über ein Netzwerk verbindet.
Direkt daran angeschlossen sind alle drei \emph{Nvidia Jetson Nanos}, der \emph{Raspberry Pi}, die \emph{\gls{mcu}} und
der nach außen verfügbar gemachte \emph{RJ-45 Port}.
Der Raspberry Pi kann als zentraler Baustein für alle Entwickler am \gls{go1} bezeichnet werden.
Bis auf dedizierte Auswertungen oder Zugriffe auf die Kameramodule werden die meisten Prozesse zumindest auf dem Pi verwaltet.
Die Nanos hingegen verarbeiten die ihnen zugeordneten Kameramodule, mit der Ausnahme des Nanos im Kopf des Roboters,
der zudem auch die Sensordaten des nach vorne gerichteten Ultraschallsensors abgreift und verfügbar macht.
Folgende Übersicht zeigt die Verteilung der zugreifbaren Recheneinheiten zu den zu verwaltenden Bausteinen.
\newline

\begin{table}[h]
    \centering
    \begin{tabular}{l|l|l|l}
        \textbf{Raspberry Pi} & \textbf{Nano 1 (Kopf)} & \textbf{Nano 2} & \textbf{Nano 3} \\ \hline
        &                        &                 &                 \\
        Wifi Modul            & \gls{led}-Steuerung    & Videoauswertung & Videoauswertung \\
        Webhosting            & Audio-Ausgabe          & links + rechts  & Rumpf unten     \\
        App-Verbindung        & Ultraschall frontal    &                 &                 \\
        Monitoring            & Videoauswertung        &                 &                 \\
        Bibliotheken          & Kopf vorne/unten       &                 &                 \\
        Ultraschall seitlich  &                        &                 &                 \\
        &                        &                 &
    \end{tabular}
    \label{tab:ueberblick_hardware_funktionen}
\end{table}
\todo{welcher Nano wo was wann, links rechts?}

Das nächste Kapitel geht auf Grundlage des Überblicks genauer auf die einzelnen Bausteine der internen Hardware des \gls{go1}
ein.

\subsubsection{Kernelemente}

Als Kernelemente des \gls{go1} werden der verbaute \emph{Raspberry Pi} und die drei verbauten \emph{Nvidia Jetson Nanos}
bezeichnet.
Grundsätzlich ist die \gls{mcu} ebenfalls als Kernelement zu bezeichnen, sie ist jedoch nicht für den Zugriff durch den
Entwickler freigeschaltet und wird deshalb nicht weiter betrachtet.
Zur genaueren Inspektion der Komponenten kann ein externer Rechner per \emph{Ethernet} an den in Kapitel \ref{subsubsec:recheneinheiten}
\emph{RJ-45}-Port angeschlossen werden.
Diesem Rechner muss dann eine statische \gls{ip}-Adresse im Netz \texttt{192.168.123.0/24} vergeben werden.
Da die \gls{ip}-Adresse noch nicht vergeben sein darf, wurde zur Analyse im Rahmen dieser Arbeit die Adresse \texttt{192.168.123.51}
verwendet.

\myparagraph{Raspberry Pi}
\label{par:raspi}

Laut Hersteller-Dokumentation ist im Roboter ein \emph{Raspberry Pi \num{4}} verbaut.
Anhand der Dokumentation erkennt man lediglich die \gls{ip}-Adresse des Pi - \texttt{192.168.123.161}, jedoch keine Informationen
zu den Eigenschaften dessen.
Zur Prüfung der Eigenschaften des Pis kann sich mit dem Roboter per Ethernet verbunden werden.
Über den Standard-Nutzer kann sich laut Dokumentation per \gls{ssh} auf den Roboter verbunden werden.

\begin{lstlisting}[language=sh, label=lst:pi-ssh]
# user: pi, password: 123, root-password:123
ssh pi@192.168.123.161
\end{lstlisting}

Als Erstes soll das genaue Modell des Raspberry Pi erkannt werden:

\begin{lstlisting}[language=sh, label=lst:pi-model]
pi@raspberrypi:~ $ grep Model /proc/cpuinfo
Model		: Raspberry Pi Compute Module 4 Rev 1.0
\end{lstlisting}

Die Prüfung der verbauten Variante des Compute Model 4 lässt dich durch folgendes Kommando durchführen:

\begin{lstlisting}[language=sh, label=lst:pi-ram]
pi@raspberrypi:~ $ grep MemTotal /proc/meminfo
MemTotal:        1894664 kB
\end{lstlisting}

Eine kurze Prüfung der Herstellerwebsite zeigt uns, dass ein \texttt{Broadcom BCM2711 quad-core Cortex-A72 (ARM v8) 64-bit SoC @ 1.5GHz}
in der Variation mit \num{2} \gls{gb} Arbeitsspeicher verbaut ist.
Die Boot-Partition und der initiale Festplattenspeicher werden bei diversen Kleinplatinenrechner oftmals über eine \gls{sd}-Karte
realisiert.
Um das auf dem Raspberry Pi zu prüfen kann die Belegung des Dateisystems ausgegeben werden.
Die temporären Dateisysteme werden hierbei ausgeschlossen.

\begin{lstlisting}[language=sh, label=lst:pi-sd]
pi@raspberrypi:~ $ df -HTx tmpfs -x devtmpfs
Filesystem     Type  Size  Used Avail Use% Mounted on
/dev/root      ext4   32G   18G   13G  59% /
/dev/mmcblk0p1 vfat  265M   69M  196M  26% /boot
\end{lstlisting}

Zu erkennen ist, dass tatsächlich eine \gls{sd}-Karte als Boot-Partition unter \texttt{/dev/mmcblk0p1} eingebunden wurde.
Zudem lässt sich die Gesamtgröße des Dateisystems ablesen - \num{32} \gls{gb}.
Zur Prüfung des verbauten Betriebssystems und des Linux Kernels kann folgendes Kommando verwendet werden:

\begin{lstlisting}[language=sh, label=lst:pi-os]
pi@raspberrypi:~ $ grep PRETTY_NAME /etc/os-release
PRETTY_NAME="Debian GNU/Linux 10 (buster)"
pi@raspberrypi:~ $ uname -r
5.4.81-rt45-v8+
\end{lstlisting}

Zusammenfassend lassen sich die Kerndaten des Pis wie in Tabelle \ref{tab:data-raspi} dargestellt:

\begin{table}[h]
    \centering
    \begin{tabular}{|r|l|}
        \hline
        Modell   & Raspberry Pi Compute Module 4 Rev 1.0\\ \hline
        SoC      & Broadcom BCM2711 quad-core Cortex-A72 (ARM v8) 64-bit SoC @ 1.5GHz\\ \hline
        RAM      & \num{1894664} kB (\num{1,8} \gls{gb}) Arbeitsspeicher\\ \hline
        Speicher & \num{32} \gls{gb} Festplattenspeicher über eine \gls{sd}-Karte\\ \hline
        OS       & Debian 10 (Buster)\\ \hline
        Kernel   & Linux Kernel 5.4.81-rt45-v8+ \\ \hline
    \end{tabular}\caption{Kenndaten des Raspberry Pi}\label{tab:data-raspi}
\end{table}

\todo[inline]{Angeschlossene Hardware}
\todo[inline]{Beschreibung des Home Ordners}

\myparagraph{Nvidia Jetson Nano Kopf}
\label{par:nano-kopf}

Folgt man der Dokumentation der Hersteller erreicht man die Recheneinheit im Kopf des Roboters unter der \gls{ip} Adresse \texttt{192.168.123.13}.
Verbinden lässt sich der Rechner über den Nutzer \texttt{unitree} und dem Passwort \texttt{123}.

\begin{lstlisting}[language=sh, label=lst:nanos-ssh]
# user: unitree, password: 123, root disabled
ssh unitree@192.168.123.<nano-ip(13|14|15)>
\end{lstlisting}

Laut Hersteller ist auf allen drei Nvidia Chips das Betriebssystem Ubuntu installiert, welches auf Debian basiert, aber einige nützliche
Funktionen über die Basis von Debian hinaus mitbringt.
So auch den Befehl \texttt{lshw}, über den sich eine Zusammenfassung der auf dem System verwendeten Hardware ausgeben lässt.

\begin{lstlisting}[language=sh, label=lst:nanos-hardware-kopf, columns=fixed]
unitree@unitree-desktop:~$ sudo lshw -short
[...] Class       Description
[...] =======================
[...] system      NVIDIA Jetson Nano Developer Kit
[...] memory      3962MiB System memory
[...] bridge      NVIDIA Corporation
[...] multimedia  USB2.0 Camera RGB
[...] multimedia  USB2.0 Camera RGB
[...] generic     CP2102N USB to UART Bridge Controller
[...] multimedia  USB Audio Device
\end{lstlisting}

Erkennbar ist über die gekürzte Ausgabe, dass im Kopf des \gls{go1} ein \emph{NVIDIA Jetson Nano} mit \num{4} \gls{gb}
Arbeitsspeicher verbaut ist.
Zudem sind per \gls{usb} vier externe Geräte angeschlossen, ein Lautsprecher im Rücken des Kopfes, ein Bridge-Controller
zur Steuerung der beiden \gls{led}-Bänder und zwei Kameras.
Die nach vorne gerichtete Kamera ist unter \texttt{/dev/video1} gemountet, die im Kopf nach unten gerichtete Kamera unter
\texttt{/dev/video0}.
Unter den Mountpoints der Kameras können diese ausgelesen und als Quelle verwendet werden.
Ein Blick auf die Dateisysteme zeigt im Gegensatz zum selben Befehl auf dem Raspberry Pi lediglich die \texttt{root}-Partition,
ein weiteres Inspizieren zeigt dann jedoch ebenfalls die \texttt{boot}-Partition auf der \gls{sd}-Karte.
Die Prozessorvariante in \texttt{/proc/cpuinfo} gibt \texttt{ARMv8 Processor rev 1 (v8l)} aus.

\begin{lstlisting}[language=sh, label=lst:nanos-kopf-fs, columns=fixed]
unitree@unitree-desktop:~$ df -Hx tmpfs -x devtmpfs
Filesystem      Size  Used Avail Use% Mount
/dev/mmcblk0p1   15G   12G  2.5G  83% /

unitree@unitree-desktop:~$ sudo lsblk
NAME         FSTYPE   SIZE MOUNTPOINT
mmcblk0              14.7G
-> mmcblk0p1  ext4     14G /
mmcblk0boot0            4M
mmcblk0boot1            4M
\end{lstlisting}

Die genaue Version des Betriebssystems lässt sich analog zum Vorgehen im Raspberry Pi ermitteln:

\begin{lstlisting}[language=sh, label=lst:nanos-kopf-os]
unitree@unitree-desktop:~$ grep PRETTY_NAME /etc/os-release
PRETTY_NAME="Ubuntu 18.04.5 LTS"
unitree@unitree-desktop:~$ uname -r
4.9.201-tegra
\end{lstlisting}

Zusammenfassend lassen sich die Kerndaten des Nvidia Jetson Nanos im Kopf des Roboters wie in Tabelle \ref{tab:data-head-nano} dargestellt:

\begin{table}[h]
    \centering
    \begin{tabular}{|r|l|}
        \hline
        Modell    & NVIDIA Jetson Nano Developer Kit\\ \hline
        Prozessor & ARMv8 Processor rev 1 (v8l)\\ \hline
        RAM       & \num{3962} MiB (\num{4} \gls{gb}) Arbeitsspeicher\\ \hline
        Speicher  & \num{16} \gls{gb} Festplattenspeicher über eine \gls{sd}-Karte\\ \hline
        OS        & Ubuntu 18.04.5 LTS\\ \hline
        Kernel    & 4.9.201-tegra \\ \hline
    \end{tabular}\caption{Kenndaten des Nvidia Jetson Nano im Kopf des Roboters}\label{tab:data-head-nano}
\end{table}

\todo[inline]{Beschreibung des Home Ordners}

\myparagraph{Nvidia Jetson Nano Rumpf}
\label{par:nano-rumpf}

Der Nano im Rumpf des Roboters is dem Nano im Kopf des Roboters gegenüber bis auf die angeschlossenen Geräte identisch.
Die Ausgaben für das Betriebssystem in \texttt{/etc/os-release}, der Kernel-Version aus \texttt{uname -r} und der Informationen
zur \gls{sd}-Karte aus \texttt{df -Hx tmpfs -xdevtmpfs} stimmen bei beiden Geräten überein.
Die Unterschiede in der verbundenen Hardware geht aus folgender Ausgabe hervor:

\begin{lstlisting}[language=sh, label=lst:nanos-hardware-rumpf, columns=fixed]
unitree@unitree-desktop:~$ sudo lshw -short
[...] Class       Description
[...] =======================
[...] system      NVIDIA Jetson Nano Developer Kit
[...] memory      3964MiB System memory
[...] bridge      NVIDIA Corporation
[...] multimedia  USB2.0 Camera RGB
[...] multimedia  USB2.0 Camera RGB
\end{lstlisting}

Ähnlich zum Nano im Kopf des \gls{go1} sind zwei Kameras verbaut, eine Kamera links im Rumpf von hinten betrachtet unter dem Mounting-Point
\texttt{/dev/video1} und eine Kamera rechts im Rumpf unter \texttt{/dev/video0}.
Als Übersicht zu den Kerndaten des Nanos im Rumpf des Roboters kann die Übersicht \ref{tab:data-head-nano} verwendet werden.

\todo[inline]{Inhalte Ordner}

\myparagraph{Nvidia Jetson Xavier NX}
\label{par:nx}

Die mit Abstand

\begin{lstlisting}[language=sh, label=lst:nanos-hardware-rumpf-nx, columns=fixed]
unitree@nx:~$ sudo lshw -short
[sudo] password for unitree:
[...] Class       Description
[...] =======================
[...] system      NVIDIA Jetson Xavier NX Developer Kit
[...] memory      7773MiB System memory
[...] bridge      NVIDIA Corporation
[...] multimedia  USB2.0 Camera RGB
\end{lstlisting}


\subsubsection{Netzwerk}
\label{subsubsec:netzwerk}

Aufgrund der vielen Komponenten, Funktionen und den möglichen Erweiterungen des \gls{go1} ist eine robuste interne Kommunikation nötig.
Die interne Kommunikationsstruktur des Roboters baut großenteils auf Netzwerkstandards wie \emph{Ethernet} und \emph{Wi-Fi} auf,
setzt besonders in der Konnektivität mit externen Komponenten jedoch zusätzlich auf weitere Standards wi \emph{Bluetooth} und \emph{\gls{wwan}}.

Das folgende Kapitel erläutert die vorhandene Kommunikation der internen und externen Komponenten des \gls{go1} und analysiert diese
auf ihre Stärken und Schwächen.
Zudem soll die Methodik der Analyse des Netzwerks und mögliche Problemfeststellungen und -behebungen festgehalten werden.

\myparagraph{Überblick}
\label{par:netzwerk_ueberblick}

Abbildung~\ref{fig:netzwerk_ueberblick} gilt als Referenz für die folgenden Ausführungen.
Zentrale Einheit der Kommunikation sind der verbaute Ethernet Switch \todo{was für switch} und
der intern verbaute \emph{Raspberry Pi}.
Wie auf Abbildung~\ref{fig:netzwerk_ueberblick} zu erkennen ist, sind alle fünf Recheneinheiten -
der Raspberry Pi, die \gls{mcu}, die beiden NVIDIA Jetson Nanos und der Jetson Xavier NX -
per \emph{Ethernet} mit dem Switch verbunden.
Auch der extern zugängliche Ethernet-Port auf dem Rücken des Roboters ist mit dem Switch verbunden.

\begin{figure}[h]
    \frame{\includegraphics[width=\linewidth]{img/netzwerk/netzwerk_ueberblick}}
    \caption{Überblick über Netzwerkkonfiguration}\label{fig:netzwerk_ueberblick}
\end{figure}\todo{Überblick Quelle aus Anleitung}

Alle geswitchten Komponenten des Netzwerks sind im \texttt{192.168.123.0/24}-Netzwerk registriert.
\todo{was ist gateway? nötig?}
Dabei ist die Verteilung der \gls{ip}-Adressen folgendermaßen vorkonfiguriert:
\begin{itemize}
    \item \textbf{\gls{mcu}:} \texttt{192.168.123.10}
    \item \textbf{Raspberry Pi:} \texttt{192.168.123.10}
    \item \textbf{NVIDIA Jetson Nanos:}
    \begin{enumerate}
        \item Kopf: \texttt{192.168.123.13}
        \item Seiten: \texttt{192.168.123.14}
    \end{enumerate}
    \item \textbf{NVIDIA Jetson Xavier NX:} \texttt{192.168.123.15}
\end{itemize}
Dem Endgerät, das am externen Ethernet-Port an der Oberseite des Roboters angesteckt werden kann,
muss eine statische \gls{ip}-Adresse im Bereich \texttt{192.168.123.0/24} vergeben werden, die nicht bereits von einem der oben genannten
Geräten verwendet wird.

Der Raspberry Pi hat zusätzlich zu seiner physischen Verbindung zum Switch und der \texttt{192.168.123.161}-\gls{ip}-Addresse
noch ein \gls{wwan} Modul verbaut, mit welchem er das Netz \texttt{192.168.12.0/24} publiziert.
Dieses Netz wird ab Werk für die Verbindung der App mit dem System benötigt.
Des Weiteren kann dieses Netz genutzt werden, um eine kabellose Verbindung mit dem Gesamtsystem
des Roboters herzustellen.
Hierzu mehr in Kapitel~\ref{par:raspi}.

\todo{MCU und Bluetooth? Wie?}
\todo{Welche Art Netzwerk, switched, routed, hub?}







\subsubsection{Lokales Netzwerk}
\label{subsubsec:lokales-netzwerk}
% Wifi Wlan1 für Netzwerk
% Was wird verbreitet, wer nutzt, App etc

Der Raspberry Pi des \gls{go1} nutzt eines seiner Netzwerk-Schnittstellen, um ein \gls{wlan} zu publizieren.
Ein Blick auf das Autostartmodul \texttt{configNetwork} zeigt, dass hierfür das Interface \texttt{wlan1} verwendet wird.

\lstinputlisting[language=Bash,numbers=left,xleftmargin=\dimexpr2.5em-1pt,framexleftmargin=2em,firstline=35,lastline=38,firstnumber=35,caption={[Konfiguration des hostapd in \texttt{configNetwork.sh}]}]{listing/configNetwork.sh}

\noindent Um weitere Informationen Konfiguration des \gls{wlan} zu finden, muss die \gls{hostapd} Konfiguration in
\texttt{/etc/\allowbreak hostapd/\allowbreak hostapd\allowbreak .conf} betrachtet werden.
Der Service \texttt{\gls{hostapd}} ist ein für Nutzer eines Systems verfügbarer Service für diverse Access Points und
Authentifizierungsserver\footcite{hostapd-doc}.

\lstinputlisting[language=Bash,numbers=left,xleftmargin=\dimexpr2.5em-1pt,framexleftmargin=2em,firstline=18,lastline=22,firstnumber=18,caption={[Konfiguration des \texttt{hostapd}]}]{listing/hostapd.conf}

\noindent Hier ist ersichtlich, dass die \gls{ssid} des \gls{wlan} Access Points der Seriennummer des jeweiligen \gls{go1} entspricht.
Diese ist auf diversen Teilen des Roboters gekennzeichnet.
Das Standardpasswort ist \texttt{00000000} und kann in der Konfigurationsdatei ebenfalls geändert werden.
Nach Neustart des Interfaces \texttt{wlan1} tritt dieses neue Passwort in Kraft.

Ebenfalls hilfreich kann die Funktion einer versteckten \gls{ssid} sein, da der Access Point des \gls{go1} sonst in unmittelbarer
Nähe zum Roboter für jedermann einsichtig wäre, was ein potenzielles Sicherheitsrisiko im regulären Betrieb darstellt.
Hierfür muss lediglich die Zeile \texttt{ignore\_\allowbreak broadcast\_\allowbreak ssid=1} am Ende der Datei hinzugefügt werden.
Die Einstellung bewirkt, dass die Publizierung des \gls{wlan} keine \gls{ssid} bekannt gibt und Verbindungsanfragen ohne eine Angabe der
vollständigen \gls{ssid} ignoriert werden.


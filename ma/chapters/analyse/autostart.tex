\subsubsection{Software Autostart}
\label{subsubsec:software-autostart}

Unitree Robotics hat bei der Implementierung der Startsequenzen ihrer Softwarekomponenten eine einfache, für den Nutzer
zugängliche Autostart Funktion der Betriebssysteme des Raspberry Pis und der NVIDIA Jetsons verwendet.
Auf den Ubuntu Systemen wird das Paket \texttt{gnome\allowbreak -start\-up-\allowbreak app\-li\-cations} verwendet.
Die Installation des Paketes kann durch den Befehl \texttt{apt list --installed | grep gnome-\allowbreak start\-up-\allowbreak app\-li\-cations}
geprüft werden.
Anzumerken ist, dass die Funktionalität von der Desktop-Umgebung \emph{Gnome} bereitgestellt wird, nicht vom Betriebssystem
der Recheneinheiten.
Die auf \gls{lxde} basierende Desktop-Umgebung \gls{pixel}, die auf dem Raspberry Pi installiert ist, liefert eine andere
Möglichkeit der Autostart-Konfiguration.
Dennoch ist die Einrichtung identisch zur Bibliothek \texttt{gnome-\allowbreak start\-up-\allowbreak app\-li\-cations}, welche aber nicht auf dem System
installiert ist.
Eine offizielle Dokumentation für die Autostart-Funktion des Pis ist deshalb nicht verfügbar, es ist lediglich ersichtlich,
dass die Funktion dem auf dem Open-Source-Projekt \emph{FreeDesktop} beschriebenen Standard \emph{Autostart Of Applications During Startup}
entspricht\footcite{freedesktop_autostart}.
Die Funktion wird in diesem Kapitel exemplarisch am Raspberry Pi gezeigt.

\myparagraph{Implementierung}

Zu Prüfung der Implementierung muss sich vorerst auf den Raspberry Pi verbunden werden.
Hier Prüfen wir zunächst, welcher Benutzer auf dem System nach einem Boot automatisch eingeloggt wird, damit die
Desktop-Umgebung die passenden Programme startet.
Erwartet wird hier der Benutzer \emph{pi}, was bei der Prüfung bestätigt wird.

\begin{lstlisting}
pi@raspberrypi:~ $ cat /etc/lightdm/lightdm.conf  | grep autologin-user | grep -v \#
autologin-user=pi
\end{lstlisting}

\noindent Der im FreeDesktop Standard definierte Ordner \texttt{/home/pi/.config/autostart/} enthält lediglich eine Datei des Typs
\texttt{.desktop}.
Dateien mit dieser Endung werden von der Desktop-Umgebung verwendet, um Programme zu starten und in der Oberfläche
gegebenenfalls weitere Informationen wie Bilder und Beschreibungen anzuzeigen.

\begin{lstlisting}[language=sh]
pi@raspberrypi:~ $ ls /home/pi/.config/autostart/
unitree.desktop
pi@raspberrypi:~ $ cat /home/pi/.config/autostart/unitree.desktop
[Desktop Entry]
Name=unitree
Comment=unitree autostart
Exec=bash /home/pi/UnitreeUpgrade/start.sh
Terminal=false
Type=Application
Categories=System;Utility;Archiving;
StartupNotify=false
NoDisplay=true
\end{lstlisting}

\noindent Der Wert des Feldes \texttt{Exec} wird nach dem Boot-Vorgang und dem Einloggen des Benutzers \emph{pi} als
Kommandozeilenbefehl interpretiert und ausgeführt.
Ein Blick auf die Zeilen \num{6} und \num{7} des ausgeführten Skripts verweise lediglich auf ein weiteres Skript.

\lstinputlisting[language=Bash,firstline=6,lastline=7,numbers=left,xleftmargin=2em,framexleftmargin=1.5em,firstnumber=6]{listing/start.sh}

\noindent Das Skript \texttt{/home/\allowbreak pi/\allowbreak Unitree/\allowbreak autostart/\allowbreak update.sh}
legt eine neue Log-Datei an und initiiert dann den Autostart Prozess.

\begin{lstlisting}[language=Bash,numbers=left,xleftmargin=2.5em,framexleftmargin=2em,firstnumber=13]
for dir in `cat .startlist.sh`
do
  if [[ $dir = \#* ]] ; then
    echo $dir': skipped' >>/home/unitree/Unitree/autostart/.startlog
  else
    cd $dir
    echo $dir':'`sed -n '1p' version.txt` >> ${scriptPath}.detailedVersion
    ./$dir.sh
    sleep 3
  fi
  cd $scriptPath
done
\end{lstlisting}

\noindent Der Inhalt der Datei \texttt{.startlist.sh} ist eine durch Zeilenumbrüche getrennte Liste aller Ordner, welche
zum Autostart nach ausführbaren Skripten durchsucht werden.
In Zeile \num{13} wird über alle diese Ordner iteriert.
Zeile \num{15} prüft, ob der aktuelle Ordner in der Liste auskommentiert wurde und überspringt diesen gegebenenfalls.
In Zeile \num{18} wird andernfalls in den Ordner navigiert und in Zeile \num{20} jenes Skript in dem Ordner ausgeführt,
das denselben Namen wie der Ordner selbst inklusive der Dateiendung \texttt{.sh} hat.

Zur Erweiterung der Autostart-Funktion oder dem Hinzufügen eigener Prozesse nach Systemstart muss somit lediglich ein
Ordner im Pfad \texttt{/home/\allowbreak pi/\allowbreak Unitree/\allowbreak autostart/} angelegt werden, in dem eine
ausführbare Datei mit dem Namen des Ordners inklusive der Dateiendung \texttt{.sh} liegt.
Weitere Informationen zu dieser Vorgehensweise und ein Beispiel der Erweiterung werden in Kapitel \ref{subsubsec:bms-monitor}
gezeigt.

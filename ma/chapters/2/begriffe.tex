\subsection{Begriffe}

Zur Einführung in die tatsächlichen Inhalte der Arbeit sollen die Begriffe \emph{Robotik} beziehungsweise \emph{Roboter} und \emph{\gls{ki}} genauer erläutert werden.
Durch die Erklärung und Abgrenzung der Begriffe und ihrer Relevanz für diese Arbeit wird im Anschluss noch die Verbindung der Bereiche erläutert.

\subsubsection{Robotik}
Wie der Titel dieser Arbeit \enquote{\mytitle} erkennen lässt, handelt diese Arbeit hauptsächlich vom Umgang und dem sinnvollen Einsatz eines Roboters.
Das Fachgebiet, welches sich mit Robotern beschäftigt, bezeichnet sich trivialerweise mit dem Begriff \emph{Robotik}.

Die Definition des Wortes \emph{Roboter} ist hingegen nicht so trivial.
So beziehen sich viele wissenschaftliche Definitionen oftmals genauer auf Industrieroboter, welche den einfachen Zweck der Entlastung menschlicher Arbeitskräfte haben.
In der Gesellschaft werden Roboter jedoch oftmals als Maschinen verstanden, die dem Menschen oder anderen bereits bekannten Kreaturen sehr ähnlich sind.\footcite{grundlagen_der_robotik}
Allgemein lässt sich das Wort \emph{Roboter} vermutlich mit folgender Definition weit fassend beschreiben:

\begin{quote}
    [...] Automat, der ferngesteuert oder nach Sensorsignalen bzw.\ einprogrammierten Befehlsfolgen anstelle eines Menschen bestimmte mechanische Tätigkeiten verrichtet\footcite{duden_roboter}
\end{quote}

% Etwas Geschichte?

Geschichtlich ordnen sich die ersten Ideen, die heute mindestens entfernt der Robotik zuzuordnen sind, in die Zeit der Antike ein.
Die Idee der Maschine, welche dem Menschen einfache Arbeiten abnehmen kann oder derer, die von menschlicher Hand gefertigt den Menschen gleich ist, hat seit dem über die gesamte vergangene Zeit die Kreativen der Welt begeistert.
Von automatisierten Maschinen aus den Skizzenbüchern des \emph{Da Vinci} bis hin zu den ersten Mensch-ähnlichen Maschinen der Firma \emph{Honda} in den späten \num{1990}-ern.\footcite{grundlagen_der_robotik}
% Abschluss
Der Vielfalt der Robotik entgegen wird sich diese Arbeit außerhalb der Grundlagenbereiche ausschließlich mit vierbeinigen Robotern auseinandersetzen, welche oftmals durch ihre Ähnlichkeit mit Hunden herausstechen.

\subsubsection{Künstliche Intelligenz}


\subsubsection{Einordnung}

% Wie sind Roboter und KI in dieser Arbeit vernetzt, worin kann man sie in ihren Gebieten einordnen? Humanoid, etc.
\subsection{Konnektivität}
\label{subsec:konnektivitat}
% Erweiterung und Redundanz der Konnektivität
% Wie kann der Hund kommunizieren
% Was kann man wofür einsetzen?

In Kapitel \ref{subsubsec:netzwerk} wurde bereits erläutert, wie das interne Netzwerk des \gls{go1} und all seiner
Rechenkomponenten aufgebaut ist.
In Kapitel \ref{subsec:hardware-architektur} wurde gezeigt, wie sich auf die einzelnen Rechenkomponenten verbunden werden kann.
Im folgenden Kapitel sollen weitere Möglichkeiten erarbeitet werden, um den \gls{go1} über diverse Netzwerke zugänglich zu machen.
Alle folgenden Erweiterungen der Konnektivität wurden getestet und dementsprechend dokumentiert.

\subsubsection{Wifi}
\label{subsubsec:wifi}
% mehrere interfaces, eines im broadcasting eines im connection mode
% Verbindungen unterwegs möglich?
% Fremdes Wifi bis shutdown?

Der Raspberry Pi des \gls{go1} besitzt drei \gls{wlan}-Interfaces, von denen lediglich eines für das Spannen des eigenen
Access-Points verwendet wird.
Das Interface \texttt{wlan2} kann verwendet werden, um den Roboter innerhalb eines bereits bestehenden Netzwerkes kabellos
mit dem Internet zu verbinden.
Hierfür muss die \texttt{configNetwork} Autostart-Funktion auf dem Raspberry Pi angepasst werden.
Diese ist, wie in Kapitel \ref{subsubsec:software-autostart} dokumentiert, im Pfad
\texttt{/home/\allowbreak pi/\allowbreak Unitree/\allowbreak autostart/\allowbreak configNetwork} zu finden.
Im Skript \texttt{configNetwork.sh} werden zu Beginn alle Interfaces abgeschaltet, worauf nur die Interfaces \texttt{eth0}
und \texttt{wlan1} wieder aktiviert und konfiguriert werden.
Das Interface \texttt{wlan2} muss deshalb zunächst wieder aktiviert werden.

\begin{lstlisting}[language=Bash]
pi@raspberrypi:~ $ sudo ifconfig wlan2 up
\end{lstlisting}

\myparagraph{Verbindung durch den wpa\_supplicant}

Das installierte Betriebssystem des Pis, Debian (Raspbian) \num{10}, verwendet zur Konfiguration der kabellosen Netzwerkverbindungen
das Paket \texttt{wpa\_supplicant}.
Um die sensiblen Netzwerkkonfigurationen nicht zu korrumpieren, wird hierfür eine neue Konfigurationsdatei zum Verbinden
mit einem Access-Point erstellt.
Hierfür wird im Verzeichnis \texttt{/etc/wpa\_supplicant/} ein neues Verzeichnis für eigene Verbindungen angelegt.

\begin{lstlisting}[language=Bash]
pi@raspberrypi:~ $ mkdir /etc/wpa_supplicant/config.d
\end{lstlisting}

\noindent Danach werden die nötigen Konfigurationen in eine Datei names \texttt{wlan2.conf} geschrieben.
Wichtig ist hierbei das Setzen der korrekten Landeskürzung, um dem Betriebssystem klarzustellen, welche rechtlichen Grundlagen
für die kabellose Verbindung über \gls{wlan} eingehalten werden müssen.

\begin{lstlisting}[language=Bash]
pi@raspberrypi:~ $ echo "\
ctrl_interface=DIR=/var/run/wpa_supplicant GROUP=netdev
update_config=1
country=DE \n" \
> wlan2.conf
\end{lstlisting}

\noindent Diese Länderkürzung muss auch noch global im System hinterlegt werden.

\begin{lstlisting}[language=Bash]
iw region set DE
\end{lstlisting}

\noindent Daraufhin können die Zugangsdaten zum Access-Point verschlüsselt an die oben erstellte \texttt{wlan2.conf}-Datei angehängt werden.

\begin{lstlisting}[language=Bash]
pi@raspberrypi:~ $ wpa_passphrase Beispiel-SSID passwort >> /etc/wpa_supplicant/conf.d/wlan2.conf
\end{lstlisting}

\noindent Die Datei sollte nun wie folgt aussehen:

\begin{lstlisting}[language=Bash]
ctrl_interface=DIR=/var/run/wpa_supplicant GROUP=netdev
update_config=1
country=DE
network={
  ssid="Beispiel-SSID"
  #psk="passwort"
  psk=6d2c8604ecb1f4825d410b859ed0fa19621bea7ffa0b1c9b8bdda995c7135c20
}
\end{lstlisting}

\noindent Die auskommentierte \texttt{psk} Zeile sollte hierbei zur Geheimhaltung des Passworts entfernt werden.
Nun kann das Interface mit der erstellten Konfiguration mit dem gewünschten Access-Point verbunden werden.

\begin{lstlisting}[language=Bash]
pi@raspberrypi:~ $ wpa_supplicant -B -i wlan2 -c /etc/wpa_supplicant/conf.d/wlan2.conf
\end{lstlisting}

\noindent Sollte bereits eine laufende Konfiguration für das Interface vorhanden sein, so muss diese zuerst mit folgendem
Befehl entfernt werden.

\begin{lstlisting}[language=Bash]
pi@raspberrypi:~ $ sudo rm /var/run/wpa_supplicant/wlan2
\end{lstlisting}

Der Raspberry Pi des Roboters ist nun mit dem Access Point verbunden.
Ist dieses ebenfalls am Internet angebunden, so hat der Raspberry Pi eine funktionierende Netzwerkverbindung,
solange er sich in Funknähe des Access-Points befindet.

\myparagraph{Automatisierung der Verbindung}

Zur automatischen Verbindung des Pis mit dem Internet nach Systemstart kann die Autostart-Funktion \texttt{configNetwork}
angepasst werden.
Hierfür sollte ein weiteres Skript erstellt werden, welches die neue Logik enthält.
Dieses kann wie folgt aussehen:

\lstinputlisting[language=Bash,numbers=left,xleftmargin=2.5em,framexleftmargin=2em]{listing/connectWlan2.sh}

\noindent Das Skript kann jetzt am Ende des bereits bestehenden Skripts \texttt{configNetwork.sh} aufgerufen werden.

\lstinputlisting[language=Bash,numbers=left,xleftmargin=2.5em,framexleftmargin=2em,firstline=42,lastline=42,firstnumber=42]{listing/configNetwork.sh}

\noindent Der Raspberry Pi verbindet sich nun während dem Systemstart mit dem konfigurierten Netzwerk.

\subsubsection{GSM}
\label{subsubsec:gsm}
% maybe auf Schorsch
% Für außerhalb und wo kein Network ist

\subsubsection{Forwarding}
\label{subsubsec:forwarding}

\subsubsection{Bluetooth}
% Wofür genutzt?
% Wofür nutzbar?


\subsection{Konnektivität}
\label{subsec:konnektivitat}
% Erweiterung und Redundanz der Konnektivität
% Wie kann der Hund kommunizieren
% Was kann man wofür einsetzen?

In Kapitel \ref{subsec:netzwerk} wurde bereits erläutert, wie das interne Netzwerk des \gls{go1} und all seiner
Rechenkomponenten aufgebaut ist.
In Kapitel \ref{subsec:hardware-architektur} wurde gezeigt, wie sich auf die einzelnen Rechenkomponenten verbunden werden kann.
Im folgenden Kapitel sollen weitere Möglichkeiten erarbeitet werden, um den \gls{go1} über diverse Netzwerke zugänglich zu machen.
Alle folgenden Erweiterungen der Konnektivität wurden getestet und dementsprechend dokumentiert.

\subsubsection*{Wifi}
\label{subsubsec:wifi}
% mehrere interfaces, eines im broadcasting eines im connection mode
% Verbindungen unterwegs möglich?
% Fremdes Wifi bis shutdown?

Der Raspberry Pi des \gls{go1} besitzt drei \gls{wlan}-Interfaces, von denen lediglich eines für das Spannen des eigenen
Access-Points verwendet wird.
Das freie Interface \texttt{wlan2} kann verwendet werden, um den Roboter innerhalb eines bereits bestehenden Netzwerkes kabellos
mit dem Internet zu verbinden.
Hierfür muss die \texttt{configNetwork} Autostart-Funktion auf dem Raspberry Pi angepasst werden.
Diese ist, wie in Kapitel \ref{subsubsec:software-autostart} dokumentiert, im Pfad
\texttt{/home/\allowbreak pi/\allowbreak Unitree/\allowbreak autostart/\allowbreak configNetwork} zu finden.
Im Skript \texttt{configNetwork.sh} werden zu Beginn alle Interfaces abgeschaltet, worauf nur die Interfaces \texttt{eth0}
und \texttt{wlan1} wieder aktiviert und konfiguriert werden.
Das Interface \texttt{wlan2} muss deshalb zunächst wieder aktiviert werden.

\begin{lstlisting}
pi@raspberrypi:~ $ sudo ifconfig wlan2 up
\end{lstlisting}

\myparagraph{Verbindung durch den wpa\_supplicant}

Das installierte Betriebssystem des Pis, Debian (Raspbian) \num{10}, verwendet zur Konfiguration der kabellosen Netzwerkverbindungen
das Paket \texttt{wpa\_supplicant}.
Um die sensiblen Netzwerkkonfigurationen nicht zu korrumpieren, wird hierfür eine neue Konfigurationsdatei zum Verbinden
mit einem Access-Point erstellt.
Hierfür wird im Verzeichnis \texttt{/etc/wpa\_supplicant/} ein neues Verzeichnis für eigene Verbindungen angelegt.

\begin{lstlisting}
pi@raspberrypi:~ $ mkdir /etc/wpa_supplicant/config.d
\end{lstlisting}

\noindent Danach werden die nötigen Konfigurationen in eine Datei names \texttt{wlan2.conf} geschrieben.
Wichtig ist hierbei das Setzen der korrekten Landeskürzung, um dem Betriebssystem klarzustellen, welche rechtlichen Grundlagen
für die kabellose Verbindung über \gls{wlan} eingehalten werden müssen.

\begin{lstlisting}
pi@raspberrypi:~ $ echo "\
ctrl_interface=DIR=/var/run/wpa_supplicant GROUP=netdev
update_config=1
country=DE \n" \
> wlan2.conf
\end{lstlisting}

\noindent Diese Länderkürzung muss auch noch global im System hinterlegt werden.

\begin{lstlisting}
iw region set DE
\end{lstlisting}

\noindent Daraufhin können die Zugangsdaten zum Access-Point verschlüsselt an die oben erstellte \texttt{wlan2.conf}-Datei angehängt werden.

\begin{lstlisting}
pi@raspberrypi:~ $ wpa_passphrase Beispiel-SSID passwort >> /etc/wpa_supplicant/conf.d/wlan2.conf
\end{lstlisting}

\noindent Die Datei sollte nun wie folgt aussehen:

\begin{lstlisting}[numbers=left,xleftmargin=\dimexpr2.5em-1pt,framexleftmargin=2em]
ctrl_interface=DIR=/var/run/wpa_supplicant GROUP=netdev
update_config=1
country=DE
network={
  ssid="Beispiel-SSID"
  #psk="passwort"
  psk=6d2c8604ecb1f4825d410b859ed0f a19621bea7ffa0b1c9b8bdda995c7135c20
}
\end{lstlisting}

\noindent Zeile \num{6} - \texttt{psk} - sollte hierbei zur Geheimhaltung des Passworts entfernt werden.
Nun kann das Interface mit der erstellten Konfiguration mit dem gewünschten Access-Point verbunden werden.

\begin{lstlisting}
pi@raspberrypi:~ $ wpa_supplicant -B -i wlan2 -c /etc/wpa_supplicant/conf.d/wlan2.conf
\end{lstlisting}

\noindent Sollte bereits eine laufende Konfiguration für das Interface vorhanden sein, so muss diese zuerst mit folgendem
Befehl entfernt werden.

\begin{lstlisting}
pi@raspberrypi:~ $ sudo rm /var/run/wpa_supplicant/wlan2
\end{lstlisting}

Der Raspberry Pi des Roboters ist nun mit dem Access Point verbunden.
Ist dieses ebenfalls am Internet angebunden, so hat der Raspberry Pi eine funktionierende Netzwerkverbindung,
solange er sich in Funknähe des Access-Points befindet.

\myparagraph{Automatisierung der Verbindung}

Zur automatischen Verbindung des Pis mit dem Internet nach Systemstart kann die Autostart-Funktion \texttt{configNetwork}
angepasst werden.
Hierfür sollte ein weiteres Skript erstellt werden, welches die neue Logik enthält.
Dieses kann wie folgt aussehen:

\lstinputlisting[numbers=left,xleftmargin=\dimexpr2.5em-1pt,framexleftmargin=2em,caption={[Verbindes des \texttt{wlan2} Interface]}]{listing/connectWlan2.sh}

\noindent Das Skript kann jetzt am Ende des bereits bestehenden Skripts \texttt{configNetwork.sh} aufgerufen werden.

\lstinputlisting[numbers=left,xleftmargin=\dimexpr2.5em-1pt,framexleftmargin=2em,firstline=42,lastline=42,firstnumber=42,caption={[Aufrufen des \texttt{connectWlan2.sh} Skripts]}]{listing/configNetwork.sh}

\noindent Der Raspberry Pi verbindet sich jetzt während dem Systemstart mit dem konfigurierten Netzwerk.

\subsubsection*{Mobilfunk}
\label{subsubsec:gsm}

Die Modelle GO1 \emph{MAX} und \emph{EDU} haben laut Herstellerwerbung ein 4G/\gls{lte} Modem verbaut, welches genutzt werden
kann, um den Roboter dauerhaft mit dem Internet zu verbinden oder ihn aus der Ferne zu steuern.
Die Dokumentation des Herstellers gibt jedoch keine Hinweise zur Konfiguration oder Nutzung des Modems.
Ein Blick auf die verbundenen \gls{usb} Geräte gibt einen Hinweis auf das genaue Modem.

\begin{lstlisting}
pi@raspberrypi:~ $ lsusb | grep Quectel
Bus 001 Device 003: ID 2c7c:0125 Quectel Wireless Solutions Co., Ltd. EC25 LTE modem
\end{lstlisting}

Verbaut ist ein \emph{Quectel EC25 LTE Modem}, welches in Linuxkreisen häufig genutzt wird und deshalb auch
genutzt werden kann, ohne neue Treiber nachinstallieren zu müssen.

\myparagraph{Vorbereitung}

Zur Verbindung des Modems mit dem Mobilfunknetz ist eine \gls{sim} Karte zwingend vorausgesetzt.
Diese kann in den dafür vorgesehenen Slot auf der Oberseite des Rumpfes eingesteckt werden.
Genaueres zur Positionierung des Slots kann in Kapitel \ref{subsubsec:recheneinheiten} nachgelesen werden.
Der Slot ist für Micro-\gls{sim}-Karten der Größe \num{12} mm auf \num{15} mm geeignet.

Der Großteil der gängigen \gls{sim}-Karten benötigt zur Entsperrung eine \gls{pin}.
Dieser sollte für die vereinfachte Nutzung im Roboter deaktiviert werden.
Da diese Funktion für alle Kartenanbieter variieren kann, sollte hierfür die Dokumentation des Anbieters konsultiert werden.
Oftmals kann die \gls{pin} Funktion auch durch Endgeräte wie Smartphones deaktiviert werden.
Auch hier unterscheiden sich die Vorgehensweisen jedoch stark, weshalb nicht weiter auf die Deaktivierung der \gls{pin}
eingegangen wird.
Nach Deaktivierung kann die \gls{sim} Karte in den Slot eingefügt und der Roboter eingeschaltet werden.


\myparagraph{Analyse des Modems}

\gls{lte} Modems können in der Regel in verschiedenen Modi betrieben werden.
Je nach Modus unterscheiden sich auch die verwendeten Treiber des Gerätes.
Um mit der Konfiguration des Modems fortzufahren, muss zuerst der Modus bestimmt werden, in dem das Gerät am Pi betrieben wird.
Hierfür kann die \texttt{lsusb} Device-Nummer genutzt werden, die folgendermaßen bestimmt werden kann.

\begin{lstlisting}
pi@raspberrypi:~ $ lsusb | grep Quectel
Bus 001 Device 003: ID 2c7c:0125 Quectel Wireless Solutions Co., Ltd. EC25 LTE modem
pi@raspberrypi:~ $ lsusb -t | grep "Dev 3"
        |__ Port 3: Dev 3, If 0, Class=Vendor Specific Class, Driver=option, 480M
        |__ Port 3: Dev 3, If 1, Class=Vendor Specific Class, Driver=option, 480M
        |__ Port 3: Dev 3, If 2, Class=Vendor Specific Class, Driver=option, 480M
        |__ Port 3: Dev 3, If 3, Class=Vendor Specific Class, Driver=option, 480M
        |__ Port 3: Dev 3, If 4, Class=Vendor Specific Class, Driver=qmi_wwan, 480M
        |__ Port 3: Dev 3, If 5, Class=Audio, Driver=snd-usb-audio, 480M
        |__ Port 3: Dev 3, If 6, Class=Audio, Driver=snd-usb-audio, 480M
        |__ Port 3: Dev 3, If 7, Class=Audio, Driver=snd-usb-audio, 480M
\end{lstlisting}

\noindent Zu erkennen ist, dass das Gerät mit dem Treiber \texttt{qmi\_wwan} betrieben wird.
In diesem Modus wird das Modem als \texttt{wwan}-Interface gelistet.

\begin{lstlisting}
pi@raspberrypi:~ $ ifconfig -a | grep wwan
wwan0: flags=4098<BROADCAST,MULTICAST>  mtu 1500
\end{lstlisting}

\noindent Für die vereinfachte Nutzung des Modems im \texttt{qmi} Modus wird das Paket \texttt{libqmi-utils} nachinstalliert.
Hierfür muss der Raspberry Pi mit dem Internet verbunden sein\footnote{Siehe Kapitel \ref{subsubsec:wifi}}.

\begin{lstlisting}
pi@raspberrypi:~ $ sudo apt update && sudo apt install libqmi-utils
\end{lstlisting}

\noindent Für die Nutzung des Werkzeugs \texttt{qmicli} des Pakets \texttt{libqmi-utils} muss zuerst noch der Gerätepfad im
Dateisystem bestimmt werden.
Hierfür können die Kernel-Nachrichten zurate gezogen werden.

\begin{lstlisting}
pi@raspberrypi:~ $ dmesg | grep qmi
[    7.606540] qmi_wwan 1-1.3:1.4: cdc-wdm0: USB WDM device
[    7.682555] qmi_wwan 1-1.3:1.4 wwan0: register 'qmi_wwan' at usb-fe980000.usb-1.3, WWAN/QMI device, 8e:c2:07:55:9c:c3
[    7.686460] usbcore: registered new interface driver qmi_wwan
\end{lstlisting}

\noindent Der Gerätename ist in der ersten Zeile der Ausgabe als \texttt{cdc-wdm0} angegeben.
Das kann über eine Ausgabe der möglichen Geräte im Verzeichnis \texttt{/dev/} bestätigt werden.

\begin{lstlisting}
pi@raspberrypi:~ $ ls /dev/cdc-*
/dev/cdc-wdm0
\end{lstlisting}


\myparagraph{Vorbereitung des Modems}

Nun müssen einige Vorbereitungen getroffen werden, um das Modem mit dem Mobilfunknetz verbinden zu können.
Zuerst muss geprüft werden, ob das Modem \emph{online} ist.

\begin{lstlisting}
pi@raspberrypi:~ $ sudo qmicli -d /dev/cdc-wdm0 --dms-get-operating-mode
error: couldn't create client for the 'dms' service: CID allocation failed in the CTL client: Transaction timed out
\end{lstlisting}

\noindent Sollte wie hier die Fehlermeldung auftreten, so ist das Gerät durch einen anderen Prozess oder Service blockiert.
Ein gängiger Service, der auf das Modem zugreift, ist oft der \texttt{ModemManager.service}.
Folgender Befehl prüft, ob dieser aktiv ist.

\begin{lstlisting}
pi@raspberrypi:~ $ systemctl | grep Modem
ModemManager.service   loaded active running   Modem Manager
\end{lstlisting}

\noindent Folgender Befehl stoppt diesen.
Danach kann das Modem noch einmal per \texttt{qmicli} abgefragt werden.

\begin{lstlisting}
pi@raspberrypi:~ $ sudo systemctl stop ModemManager
pi@raspberrypi:~ $ sudo qmicli -d /dev/cdc-wdm0 --dms-get-operating-mode
[/dev/cdc-wdm0] Operating mode retrieved:
	Mode: 'online'
	HW restricted: 'no'
\end{lstlisting}

\noindent Sollte der \texttt{qmicli} Befehl immer noch fehlschlagen, so muss geprüft werden, ob ein weiterer Prozess desselben
Pakets das Modem blockiert. 
Dies kann über die Abfrage \texttt{ps aux | grep qmi} erledigt werden.
Ein möglicher blockierender Prozess muss dann über seine Prozess-ID und den Befehl \texttt{pkill} beendet werden.
Ist der angezeigte \texttt{Mode} des \texttt{qmicli} Befehls nicht \texttt{online}, so muss dieser Wert manuell gesetzt
werden.

\begin{lstlisting}
pi@raspberrypi:~ $ sudo qmicli -d /dev/cdc-wdm0 --dms-set-operating-mode='online'
[/dev/cdc-wdm0] Operating mode set successfully
\end{lstlisting}


\myparagraph{Verbindung mit Mobilfunknetz}

Als Nächstes kann das Modem in das Mobilfunknetz eingewählt werden.
Hierfür kann folgender Befehl genutzt werden.

\begin{lstlisting}
pi@raspberrypi:~ $ sudo qmicli \
    -p \ # Request to use the 'qmi-proxy' proxy
    -d /dev/cdc-wdm0 \ # Device Path
    --device-open-net='net-raw-ip|net-no-qos-header' \ # Open device with specific link protocol and QoS flags
    --wds-start-network="apn='internet',ip-type=4" \ # IPv4 und APN
    --client-no-release-cid # Do not release the CID when exiting
[/dev/cdc-wdm0] Network started
Packet data handle: '2269312560'
[/dev/cdc-wdm0] Client ID not released:
Service: 'wds'
CID: '17'
\end{lstlisting}

\noindent Als Wert für den \gls{apn} muss hier beim Anbieter direkt Auskunft eingeholt werden, da dies je nach Kartenart und Netzwerk
unterschiedlich sein kann.
Im Test war die Angabe des \gls{apn} nicht relevant, dies kann jedoch eine mögliche Fehlerquelle sein.
Ein weiterer möglicher Fehler kann sein, dass das Modem nicht für das Protokoll \texttt{raw-ip} eingestellt wurde.
In diesem Fall muss das Netzwerk-Interface gestoppt werden, das Modem konfiguriert und daraufhin wieder aktiviert werden.
Anschließend kann der \texttt{operating-mode} nochmals geprüft werden.

\begin{lstlisting}
pi@raspberrypi:~ $ sudo ip link set wwan0 down
pi@raspberrypi:~ $ echo 'Y' | sudo tee /sys/class/net/wwan0/qmi/raw_ip
Y
pi@raspberrypi:~ $ sudo ip link set wwan0 up
pi@raspberrypi:~ $ sudo qmicli -d /dev/cdc-wdm0 --wda-get-data-format
[/dev/cdc-wdm0] Successfully got data format
                   QoS flow header: no
               Link layer protocol: 'raw-ip'
    [...]
\end{lstlisting}

\noindent Die Ausgabe zeigt ebenfalls, dass kein \texttt{QoS}-Header gesetzt werden sollte.
Der Befehl zu Verbindung mit dem Netzwerk kann nun erneut ausgeführt werden.

Als Letztes muss dem Modem im Mobilfunknetz noch eine \gls{ip} Adresse zugewiesen werden.
Sobald das Modem im Mobilfunknetz eingewählt ist, kann es über \gls{dhcp} eine \gls{ip} Adresse zugewiesen bekommen.
Hierfür wird ein weiteres Paket installiert, das eine Konfiguration sucht und - falls keine gefunden wird - eine
Anfrage im Netz zur Zuweisung einer \gls{ip} stellt.

\begin{lstlisting}
pi@raspberrypi:~ $ sudo apt install udhcpc
pi@raspberrypi:~ $ sudo udhcpc -q -f -i wwan0
udhcpc: started, v1.30.1
No resolv.conf for interface wwan0.udhcpc
udhcpc: sending discover
udhcpc: sending select for 10.114.75.205
udhcpc: lease of 10.114.75.205 obtained, lease time 7200
\end{lstlisting}

\noindent Das Modem ist jetzt mit dem Mobilfunknetz und somit mit dem Internet verbunden.

\myparagraph{Automatische Verbindung}

Mit der Konfiguration des Modems ist der Raspberry Pi bis zum nächsten Systemstart mit dem Internet verbunden.
Startet man den Pi jedoch neu, so muss die Prozedur seit Anfang des Kapitels noch einmal durchführen.
Zur vereinfachten automatisierten Verbindung mit dem Mobilfunknetz kann die in Kapitel \ref{subsubsec:software-autostart}
beschriebene Autostart-Funktion verwendet werden.
Genauer kann hier die \texttt{configNetwork} Funktion erweitert werden, wie bereits in Kapitel \ref{subsubsec:wifi}
dokumentiert.
Hierfür muss lediglich ein weiteres Skript am Ende der \texttt{configNetwork.sh} Datei aufgerufen werden.

\begin{lstlisting}[numbers=left,xleftmargin=\dimexpr2.5em-1pt,framexleftmargin=2em,firstnumber=43]
sudo ./connectWwan0.sh&
\end{lstlisting}

\noindent Der Inhalt der Datei spiegelt die Verbindung des Modems mit dem Netzwerk und das Abgreifen der \gls{ip}
über \gls{dhcp} wieder und sieht folgendermaßen aus.

\lstinputlisting[numbers=left,xleftmargin=\dimexpr2.5em-1pt,framexleftmargin=2em,caption={[Modemkonfiguration des \texttt{wwan0} Interfaces]}]{listing/connectWwan0.sh}

\noindent Der Roboter verbindet sich nun bei Systemstart per Modem mit dem Mobilfunknetz des Anbieters, vorausgesetzt,
es befindet sich eine gültige und entsperrte \gls{sim} Karte im Slot auf der Oberseite des Rumpfes.

\myparagraph{Ausblick}

Die Verbindung kann in Zukunft genutzt werden, um den Roboter aus weiter Entfernung steuern zu können.
Hierfür kann ein sogenanntes \gls{vpn} verwendet werden.
Der Raspberry Pi des Roboters müsste hierfür als \emph{Client} im \gls{vpn} registriert werden.
Verbindet man sich nun mit einem weiteren Client mit dem VPN, so sind beide Geräte im gleichen Netzwerk, unabhängig
davon, über welches Netzwerk sie mit dem Internet verbunden sind.

